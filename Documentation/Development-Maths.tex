% !TeX spellcheck = en_US
% !TeX root = DynELA.tex
%
% LaTeX source file of DynELA FEM Code
%
% (c) by Olivier Pantalé 2020
%
\chapter{DynELA Maths library}

\startcontents[chapters]
\printmyminitoc[2]\LETTRINE{T}he \DynELA~is an Explicit FEM code written in \Cpp~using a Python's interface for creating the Finite Element Models. 

\section{The Tensor2 library}

The Tensor2 library is used to store second order tensors defined in the \DynELA.

A second order tensor is a like a matrix with the following form:
\begin{equation}
T=\left[\begin{array}{ccc}
  T_{11} & T_{12} & T_{13}\\
  T_{21} & T_{22} & T_{23}\\
  T_{31} & T_{32} & T_{33}
  \end{array}\right]
\end{equation}


%@DOC:Tensor2::Tensor2()
%Warning :
%This area is an automatic documentation generated from the DynELA source code.
%Do not change anything in this latex file between this position and the @END keyword.
\textcolor{purple}{\textbf{Tensor2::Tensor2(~)}}\label{Tensor2::Tensor2()} : Default constructor of the Tensor2 class.\index[DL]{Tensor2!Tensor2(~)}\\ \hspace*{5mm}$\hookrightarrow$ Tensor2

All components are initialized to zero by default.
\begin{equation*}
T=\left[\begin{array}{ccc}
0&0&0\\
0&0&0\\
0&0&0
\end{array}\right]
\end{equation*}
%@END

%@DOC:Tensor2::Tensor2(const Tensor2)
%Warning :
%This area is an automatic documentation generated from the DynELA source code.
%Do not change anything in this latex file between this position and the @END keyword.
\textcolor{purple}{\textbf{Tensor2::Tensor2(const Tensor2)}}\label{Tensor2::Tensor2(const Tensor2)} : Copy constructor of the Tensor2 class.\index[DL]{Tensor2!Tensor2(const Tensor2)}\\ \hspace*{5mm}$\hookrightarrow$ Tensor2

%@END

%@DOC:Tensor2::~Tensor2()
%Warning :
%This area is an automatic documentation generated from the DynELA source code.
%Do not change anything in this latex file between this position and the @END keyword.
\textcolor{purple}{\textbf{Tensor2::$\sim$Tensor2(~)}}\label{Tensor2::~Tensor2()} : Destructor of the Tensor2 class.\index[DL]{Tensor2!$\sim$Tensor2(~)}

%@END

%@DOC:Tensor2::Tensor2(double,...)
%Warning :
%This area is an automatic documentation generated from the DynELA source code.
%Do not change anything in this latex file between this position and the @END keyword.
\textcolor{purple}{\textbf{Tensor2::Tensor2(double,...)}}\label{Tensor2::Tensor2(double,...)} : Constructor of the Tensor2 class.\index[DL]{Tensor2!Tensor2(double,...)}\\ \hspace*{5mm}$\hookrightarrow$ Tensor2

Constructor of a second order tensor with initialization of the 9 values.
%@END

%@DOC:Tensor2::setToUnity()
%Warning :
%This area is an automatic documentation generated from the DynELA source code.
%Do not change anything in this latex file between this position and the @END keyword.
\textcolor{purple}{\textbf{Tensor2::setToUnity(~)}}\label{Tensor2::setToUnity()} : Unity tensor.\index[DL]{Tensor2!setToUnity(~)}

\hspace*{10mm}\textcolor{red}{\textbf{Warning :} This method modifies its own argument}

This method transforms the current tensor to a unity tensor.
%@END

%@DOC:Tensor2::setToZero()
%Warning :
%This area is an automatic documentation generated from the DynELA source code.
%Do not change anything in this latex file between this position and the @END keyword.
\textcolor{purple}{\textbf{Tensor2::setToZero(~)}}\label{Tensor2::setToZero()} : Sets all components of the tensor to zero.\index[DL]{Tensor2!setToZero(~)}

\hspace*{10mm}\textcolor{red}{\textbf{Warning :} This method modifies its own argument}

%@END

\section{The Tensor3 library}

The Tensor3 library is used to store third order tensors defined in the \DynELA.

%@DOC:Tensor3::Tensor3()
%Warning :
%This area is an automatic documentation generated from the DynELA source code.
%Do not change anything in this latex file between this position and the @END keyword.
\textcolor{purple}{\textbf{Tensor3::Tensor3(~)}}\label{Tensor3::Tensor3()} : This method is the default constructor of a third order tensor.\index[DL]{Tensor3!Tensor3(~)}\\ \hspace*{5mm}$\hookrightarrow$ Tensor3

All components are initialized to zero by default.
%@END

%@DOC:Tensor3::~Tensor3()
%Warning :
%This area is an automatic documentation generated from the DynELA source code.
%Do not change anything in this latex file between this position and the @END keyword.
\textcolor{purple}{\textbf{Tensor3::$\sim$Tensor3(~)}}\label{Tensor3::~Tensor3()} : Destructor of the Tensor3 class.\index[DL]{Tensor3!$\sim$Tensor3(~)}

%@END
