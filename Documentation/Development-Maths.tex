% !TeX spellcheck = en_US
% !TeX root = DynELA.tex
%
% LaTeX source file of DynELA FEM Code
%
% (c) by Olivier Pantalé 2020
%
\chapter{DynELA Maths library}

\startcontents[chapters]
\printmyminitoc[2]\LETTRINE{T}he \DynELA~is an Explicit FEM code written in \Cpp~using a Python's interface for creating the Finite Element Models. 

\section{The Tensor2 library}

%@DOC:Tensor2::Tensor2
%Warning :
%This area is an automatic documentation generated from the DynELA source code.
%Do not change anything in this latex file between this position and the @END keyword.
\textcolor{purple}{\textbf{Tensor2::Tensor2}}\label{Tensor2::Tensor2}\index[DL]{Tensor2!Tensor2}\\
Second order tensor class.

The Tensor2 library is used to store second order tensors defined in the \DynELA. A second order tensor is a like a matrix with the following form:
\begin{equation*}
T=\left[\begin{array}{ccc}
  T_{11} & T_{12} & T_{13}\\
  T_{21} & T_{22} & T_{23}\\
  T_{31} & T_{32} & T_{33}
  \end{array}\right]
\end{equation*}
Concerning the internal storage of data, the Tensor2 data is stored in a vector of 9 components named \_data using the following storage scheme:
\begin{equation*}
T=\left[\begin{array}{ccc}
    T_{0} & T_{1} & T_{2}\\
    T_{3} & T_{4} & T_{5}\\
    T_{6} & T_{7} & T_{8}
    \end{array}\right]
\end{equation*}
%@END

\subsection{Constructor and destructor}
%@DOC:Tensor2::Tensor2()
%Warning :
%This area is an automatic documentation generated from the DynELA source code.
%Do not change anything in this latex file between this position and the @END keyword.
\textcolor{purple}{\textbf{Tensor2::Tensor2(~)}}\label{Tensor2::Tensor2()}\index[DL]{Tensor2!Tensor2(~)}\\
Default constructor of the Tensor2 class.\\ \hspace*{10mm}$\hookrightarrow$ Tensor2

All components are initialized to zero by default.
\begin{equation*}
\T=\left[\begin{array}{ccc}
0&0&0\\
0&0&0\\
0&0&0
\end{array}\right]
\end{equation*}
%@END

%@DOC:Tensor2::Tensor2(double,...)
%Warning :
%This area is an automatic documentation generated from the DynELA source code.
%Do not change anything in this latex file between this position and the @END keyword.
\textcolor{purple}{\textbf{Tensor2::Tensor2(double,...)}}\label{Tensor2::Tensor2(double,...)}\index[DL]{Tensor2!Tensor2(double,...)}\\
Constructor of the Tensor2 class.\\ \hspace*{10mm}$\hookrightarrow$ Tensor2

\begin{tcolorbox}[width=\textwidth,myArgs,tabularx={ll|R},title=Arguments of Tensor2::Tensor2]
double&t1&Component $t_{11}$ of the tensor.\\
double&t2&Component $t_{12}$ of the tensor.\\
double&t3&Component $t_{13}$ of the tensor.\\
double&t4&Component $t_{21}$ of the tensor.\\
double&t5&Component $t_{22}$ of the tensor.\\
double&t6&Component $t_{23}$ of the tensor.\\
double&t7&Component $t_{31}$ of the tensor.\\
double&t8&Component $t_{32}$ of the tensor.\\
double&t9&Component $t_{33}$ of the tensor.
\end{tcolorbox}

Constructor of a second order tensor with explicit initialization of the 9 components of the tensor.
%@END

%@DOC:Tensor2::Tensor2(Tensor2 T)
%Warning :
%This area is an automatic documentation generated from the DynELA source code.
%Do not change anything in this latex file between this position and the @END keyword.
\textcolor{purple}{\textbf{Tensor2::Tensor2(Tensor2 T)}}\label{Tensor2::Tensor2(Tensor2 T)}\index[DL]{Tensor2!Tensor2(Tensor2 T)}\\
Copy constructor of the Tensor2 class.\\ \hspace*{10mm}$\hookrightarrow$ Tensor2

\begin{tcolorbox}[width=\textwidth,myArgs,tabularx={ll|R},title=Arguments of Tensor2::Tensor2]
SymTensor2&T&Tensor to copy.
\end{tcolorbox}

%@END

%@DOC:Tensor2::operator=(SymTensor2 T)
%Warning :
%This area is an automatic documentation generated from the DynELA source code.
%Do not change anything in this latex file between this position and the @END keyword.
\textcolor{purple}{\textbf{Tensor2::operator=(SymTensor2 T)}}\label{Tensor2::operator=(SymTensor2 T)}\index[DL]{Tensor2!operator=(SymTensor2 T)}\\
Copy a SymTensor2 into a Tensor2.\\ \hspace*{10mm}$\hookrightarrow$ Tensor2

\begin{tcolorbox}[width=\textwidth,myArgs,tabularx={ll|R},title=Arguments of Tensor2::operator=]
SymTensor2&T&Symmetric tensor 2 to copy.
\end{tcolorbox}

%@END

%@DOC:Tensor2::~Tensor2()
%Warning :
%This area is an automatic documentation generated from the DynELA source code.
%Do not change anything in this latex file between this position and the @END keyword.
\textcolor{purple}{\textbf{Tensor2::$\sim$Tensor2(~)}}\label{Tensor2::~Tensor2()}\index[DL]{Tensor2!$\sim$Tensor2(~)}\\
Destructor of the Tensor2 class.

%@END

\subsection{Basic operations}

%@DOC:Tensor2::setToZero()
%Warning :
%This area is an automatic documentation generated from the DynELA source code.
%Do not change anything in this latex file between this position and the @END keyword.
\textcolor{purple}{\textbf{Tensor2::setToZero(~)}}\label{Tensor2::setToZero()}\index[DL]{Tensor2!setToZero(~)}\\
Sets all components of the tensor to zero.

\hspace*{10mm}\textcolor{red}{\textbf{Warning :} This method modifies its own argument}

\begin{equation*}
\T=\left[\begin{array}{ccc}
0&0&0\\
0&0&0\\
0&0&0
\end{array}\right]
\end{equation*}
%@END

%@DOC:Tensor2::setToUnity()
%Warning :
%This area is an automatic documentation generated from the DynELA source code.
%Do not change anything in this latex file between this position and the @END keyword.
\textcolor{purple}{\textbf{Tensor2::setToUnity(~)}}\label{Tensor2::setToUnity()}\index[DL]{Tensor2!setToUnity(~)}\\
Unity tensor.

\hspace*{10mm}\textcolor{red}{\textbf{Warning :} This method modifies its own argument}

This method transforms the current tensor to a unity tensor.
\begin{equation*}
\T=\left[\begin{array}{ccc}
1&0&0\\
0&1&0\\
0&0&1
\end{array}\right]
\end{equation*}
%@END

%@DOC:Tensor2::getTranspose()
%Warning :
%This area is an automatic documentation generated from the DynELA source code.
%Do not change anything in this latex file between this position and the @END keyword.
\textcolor{purple}{\textbf{Tensor2::getTranspose(~)}}\label{Tensor2::getTranspose()}\index[DL]{Tensor2!getTranspose(~)}\\
Transpose of a second order tensor.\\ \hspace*{10mm}$\hookrightarrow$ Tensor2

This method defines the transpose of a second second order tensor.
The result of this operation is a second order tensor defined by the following equation:
\begin{equation*}
\A=\B^T =\left[\begin{array}{ccc}
  B_{11} & B_{21} & B_{31}\\
  B_{12} & B_{22} & B_{32}\\
  B_{13} & B_{23} & B_{33}
  \end{array}\right]
\end{equation*}
%@END

%@DOC:Tensor2::operator=(double val)
%Warning :
%This area is an automatic documentation generated from the DynELA source code.
%Do not change anything in this latex file between this position and the @END keyword.
\textcolor{purple}{\textbf{Tensor2::operator=(double val)}}\label{Tensor2::operator=(double val)}\index[DL]{Tensor2!operator=(double val)}\\
Fill a second order tensor with a scalar value.\\ \hspace*{10mm}$\hookrightarrow$ Tensor2

\begin{tcolorbox}[width=\textwidth,myArgs,tabularx={ll|R},title=Arguments of Tensor2::operator=]
double&val&Value to use for the operation.
\end{tcolorbox}

This method is a surdefinition of the = operator for the second order tensor class.
\begin{equation*}
\T=\left[\begin{array}{ccc}
m&m&m\\
m&m&m\\
m&m&m
\end{array}\right]
\end{equation*}
%@END

%@DOC:Tensor2::rowSum()
%Warning :
%This area is an automatic documentation generated from the DynELA source code.
%Do not change anything in this latex file between this position and the @END keyword.
\textcolor{purple}{\textbf{Tensor2::rowSum(~)}}\label{Tensor2::rowSum()}\index[DL]{Tensor2!rowSum(~)}\\
Sum of the rows of a second order tensor.\\ \hspace*{10mm}$\hookrightarrow$ Vec3D

This method returns a vector by computing the sum of the components on all rows of a second second order tensor.
The result of this operation is a vector defined by:
\begin{equation*}
v_{i}=\sum_{j=1}^{3} T_{ji}
\end{equation*}
%@END

%@DOC:Tensor2::columnSum()
%Warning :
%This area is an automatic documentation generated from the DynELA source code.
%Do not change anything in this latex file between this position and the @END keyword.
\textcolor{purple}{\textbf{Tensor2::columnSum(~)}}\label{Tensor2::columnSum()}\index[DL]{Tensor2!columnSum(~)}\\
Sum of the columns of a second order tensor.\\ \hspace*{10mm}$\hookrightarrow$ Vec3D

This method returns a vector by computing the sum of the components on all columns of a second second order tensor.
The result of this operation is a vector defined by:
\begin{equation*}
v_{i}=\sum_{j=1}^{3}T_{ij}
\end{equation*}
%@END

%@DOC:Tensor2::getRow(short row)
%Warning :
%This area is an automatic documentation generated from the DynELA source code.
%Do not change anything in this latex file between this position and the @END keyword.
\textcolor{purple}{\textbf{Tensor2::getRow(short row)}}\label{Tensor2::getRow(short row)}\index[DL]{Tensor2!getRow(short row)}\\
Extraction of a row from a second order tensor.\\ \hspace*{10mm}$\hookrightarrow$ Vec3D

\begin{tcolorbox}[width=\textwidth,myArgs,tabularx={ll|R},title=Arguments of Tensor2::getRow]
short&row&Row to extract
\end{tcolorbox}

This method returns a vector as part of a second second order tensor.
The result of this operation with the argument j is a vector defined by:
\begin{equation*}
v_{i} = T_{ij}
\end{equation*}
%@END

%@DOC:Tensor2::getColumn(short col)
%Warning :
%This area is an automatic documentation generated from the DynELA source code.
%Do not change anything in this latex file between this position and the @END keyword.
\textcolor{purple}{\textbf{Tensor2::getColumn(short col)}}\label{Tensor2::getColumn(short col)}\index[DL]{Tensor2!getColumn(short col)}\\
Extraction of a column from a second order tensor.\\ \hspace*{10mm}$\hookrightarrow$ Vec3D

\begin{tcolorbox}[width=\textwidth,myArgs,tabularx={ll|R},title=Arguments of Tensor2::getColumn]
short&col&Column to extract
\end{tcolorbox}

This method returns a vector as part of a second second order tensor.
The result of this operation with the argument j is a vector defined by:
\begin{equation*}
v_{i} = T_{ji}
\end{equation*}
%@END

%@DOC:Tensor2::minValue()
%Warning :
%This area is an automatic documentation generated from the DynELA source code.
%Do not change anything in this latex file between this position and the @END keyword.
\textcolor{purple}{\textbf{Tensor2::minValue(~)}}\label{Tensor2::minValue()}\index[DL]{Tensor2!minValue(~)}\\
Minimum component in a second order tensor.\\ \hspace*{10mm}$\hookrightarrow$ double

This method returns the minimum component in a second second order tensor.
%@END

%@DOC:Tensor2::minAbsoluteValue()
%Warning :
%This area is an automatic documentation generated from the DynELA source code.
%Do not change anything in this latex file between this position and the @END keyword.
\textcolor{purple}{\textbf{Tensor2::minAbsoluteValue(~)}}\label{Tensor2::minAbsoluteValue()}\index[DL]{Tensor2!minAbsoluteValue(~)}\\
Minimum absolute component in a second order tensor.\\ \hspace*{10mm}$\hookrightarrow$ double

This method returns the minimum absolute component in a second second order tensor.
%@END

%@DOC:Tensor2::maxValue()
%Warning :
%This area is an automatic documentation generated from the DynELA source code.
%Do not change anything in this latex file between this position and the @END keyword.
\textcolor{purple}{\textbf{Tensor2::maxValue(~)}}\label{Tensor2::maxValue()}\index[DL]{Tensor2!maxValue(~)}\\
Maximum component in a second order tensor.\\ \hspace*{10mm}$\hookrightarrow$ double

This method returns the maximum component in a second second order tensor.
%@END

%@DOC:Tensor2::maxAbsoluteValue()
%Warning :
%This area is an automatic documentation generated from the DynELA source code.
%Do not change anything in this latex file between this position and the @END keyword.
\textcolor{purple}{\textbf{Tensor2::maxAbsoluteValue(~)}}\label{Tensor2::maxAbsoluteValue()}\index[DL]{Tensor2!maxAbsoluteValue(~)}\\
Maximum absolute component in a second order tensor.\\ \hspace*{10mm}$\hookrightarrow$ double

This method returns the maximum absolute component in a second second order tensor.
%@END

%@DOC:Tensor2::getTrace()
%Warning :
%This area is an automatic documentation generated from the DynELA source code.
%Do not change anything in this latex file between this position and the @END keyword.
\textcolor{purple}{\textbf{Tensor2::getTrace(~)}}\label{Tensor2::getTrace()}\index[DL]{Tensor2!getTrace(~)}\\
Returns the trace of a second order tensor.\\ \hspace*{10mm}$\hookrightarrow$ double

  This method returns the trace of a second order tensor, i.e. the sum of all the terms of the diagonal:
\begin{equation*}
v = tr[\T] = T_{11}+T_{22}+T_{33}
\end{equation*}
%@END

%@DOC:Tensor2::getThirdTrace()
%Warning :
%This area is an automatic documentation generated from the DynELA source code.
%Do not change anything in this latex file between this position and the @END keyword.
\textcolor{purple}{\textbf{Tensor2::getThirdTrace(~)}}\label{Tensor2::getThirdTrace()}\index[DL]{Tensor2!getThirdTrace(~)}\\
Returns the average value of the trace of a second order tensor.\\ \hspace*{10mm}$\hookrightarrow$ double

This method returns average value of the trace of a second order tensor, i.e. the sum of all the terms of the diagonal divided by 3:
\begin{equation*}
v = \frac{1}{3} tr[\T] =  \frac{1}{3} \left( T_{11}+T_{22}+T_{33} \right)
\end{equation*}
%@END

\subsection{Specific operations}

%@DOC:Tensor2::singleProduct(Tensor2 T)
%Warning :
%This area is an automatic documentation generated from the DynELA source code.
%Do not change anything in this latex file between this position and the @END keyword.
\textcolor{purple}{\textbf{Tensor2::singleProduct(Tensor2 T)}}\label{Tensor2::singleProduct(Tensor2 T)}\index[DL]{Tensor2!singleProduct(Tensor2 T)}\\
Contracted product of two second order tensors.\\ \hspace*{10mm}$\hookrightarrow$ Tensor2

\begin{tcolorbox}[width=\textwidth,myArgs,tabularx={ll|R},title=Arguments of Tensor2::singleProduct]
Tensor2&T&Second Tensor for the multiplication operation.
\end{tcolorbox}

This method defines a single contracted product of two second order tensors.
The result of this operation is also a second order tensor defined by:
\begin{equation*}
\T = \A \cdot \B
\end{equation*}
where $\A$ and $\B$ are two second order tensors.
%@END

%@DOC:Tensor2::singleProduct()
%Warning :
%This area is an automatic documentation generated from the DynELA source code.
%Do not change anything in this latex file between this position and the @END keyword.
\textcolor{purple}{\textbf{Tensor2::singleProduct(~)}}\label{Tensor2::singleProduct()}\index[DL]{Tensor2!singleProduct(~)}\\
Contracted product of a second order tensor by itself.\\ \hspace*{10mm}$\hookrightarrow$ Tensor2

This method defines a single contracted product of of a second order tensor by itself.
The result of this operation is also a second order tensor defined by:
\begin{equation*}
\T = \A \cdot \A
\end{equation*}
where $\A$ is a two second order tensor.
%@END

%@DOC:Tensor2::singleProductTxN()
%Warning :
%This area is an automatic documentation generated from the DynELA source code.
%Do not change anything in this latex file between this position and the @END keyword.
\textcolor{purple}{\textbf{Tensor2::singleProductTxN(~)}}\label{Tensor2::singleProductTxN()}\index[DL]{Tensor2!singleProductTxN(~)}\\
Contracted product of a second order tensor by its transpose.\\ \hspace*{10mm}$\hookrightarrow$ SymTensor2

This method defines a single contracted product of two second order tensors.
The result of this operation is also a second order tensor defined by:
\begin{equation*}
\T = \A^T\cdot \A
\end{equation*}
where $\A$ is a second order tensor. Result is a symmetric second order tensor.
%@END

%@DOC:Tensor2::singleProductNxT()
%Warning :
%This area is an automatic documentation generated from the DynELA source code.
%Do not change anything in this latex file between this position and the @END keyword.
\textcolor{purple}{\textbf{Tensor2::singleProductNxT(~)}}\label{Tensor2::singleProductNxT()}\index[DL]{Tensor2!singleProductNxT(~)}\\
Contracted product of a second order tensor by its transpose.\\ \hspace*{10mm}$\hookrightarrow$ SymTensor2

This method defines a single contracted product of two second order tensors.
The result of this operation is also a second order tensor defined by:
\begin{equation*}
\T = \A \cdot \A^T
\end{equation*}
where $\A$ is a second order tensor. Result is a symmetric second order tensor.
%@END

%@DOC:Tensor2::operator*(Tensor2 T)
%Warning :
%This area is an automatic documentation generated from the DynELA source code.
%Do not change anything in this latex file between this position and the @END keyword.
\textcolor{purple}{\textbf{Tensor2::operator*(Tensor2 T)}}\label{Tensor2::operator*(Tensor2 T)}\index[DL]{Tensor2!operator*(Tensor2 T)}\\
Multiplication of 2 second order tensors.\\ \hspace*{10mm}$\hookrightarrow$ Tensor2

\begin{tcolorbox}[width=\textwidth,myArgs,tabularx={ll|R},title=Arguments of Tensor2::operator*]
Tensor2&T&Second Tensor for the multiplication operation.
\end{tcolorbox}

This method defines a single contracted product of two second order tensors.
The result of this operation is also a second order tensor defined by:
\begin{equation*}
\T = \A \cdot \B
\end{equation*}
where $\A$ and $\B$ are two second order tensors.
%@END

%@DOC:Tensor2::operator*(SymTensor2 T)
%Warning :
%This area is an automatic documentation generated from the DynELA source code.
%Do not change anything in this latex file between this position and the @END keyword.
\textcolor{purple}{\textbf{Tensor2::operator*(SymTensor2 T)}}\label{Tensor2::operator*(SymTensor2 T)}\index[DL]{Tensor2!operator*(SymTensor2 T)}\\
Multiplication of 2 second order tensors.\\ \hspace*{10mm}$\hookrightarrow$ Tensor2

\begin{tcolorbox}[width=\textwidth,myArgs,tabularx={ll|R},title=Arguments of Tensor2::operator*]
Tensor2&T&Second Tensor for the multiplication operation.
\end{tcolorbox}

This method defines a single contracted product of two second order tensors.
The result of this operation is also a second order tensor defined by:
\begin{equation*}
\T = \A \cdot \B
\end{equation*}
where $\A$ is a second order tensor and $\B$ is a symmetric second order tensor.
%@END

%@DOC:Tensor2::doubleProduct(Tensor2 T)
%Warning :
%This area is an automatic documentation generated from the DynELA source code.
%Do not change anything in this latex file between this position and the @END keyword.
\textcolor{purple}{\textbf{Tensor2::doubleProduct(Tensor2 T)}}\label{Tensor2::doubleProduct(Tensor2 T)}\index[DL]{Tensor2!doubleProduct(Tensor2 T)}\\
Double contracted product of 2 second order tensors.\\ \hspace*{10mm}$\hookrightarrow$ double

\begin{tcolorbox}[width=\textwidth,myArgs,tabularx={ll|R},title=Arguments of Tensor2::doubleProduct]
Tensor2&T&Second Tensor for the multiplication operation.
\end{tcolorbox}

This method defines a double contracted product of two second order tensors.
The result of this operation is a scalar defined by:
\begin{equation*}
s = \A : \B = \sum_{i=1}^{3} \sum_{j=1}^{3} A_{ij}\times B_{ij}
\end{equation*}
where $\A$ and $\B$ are two second order tensors.
%@END

%@DOC:Tensor2::doubleProduct()
%Warning :
%This area is an automatic documentation generated from the DynELA source code.
%Do not change anything in this latex file between this position and the @END keyword.
\textcolor{purple}{\textbf{Tensor2::doubleProduct(~)}}\label{Tensor2::doubleProduct()}\index[DL]{Tensor2!doubleProduct(~)}\\
Double contracted product of a second order tensor by itself.\\ \hspace*{10mm}$\hookrightarrow$ double

This method defines a double contracted product of a second order tensor by itself.
The result of this operation is a scalar defined by:
\begin{equation*}
s = \A : \A = \sum_{i=1}^{3} \sum_{j=1}^{3} A_{ij}\times A_{ij}
\end{equation*}
where $\A$ is a second order tensor.
%@END

%@DOC:Tensor2::operator*(Vec3D V)
%Warning :
%This area is an automatic documentation generated from the DynELA source code.
%Do not change anything in this latex file between this position and the @END keyword.
\textcolor{purple}{\textbf{Tensor2::operator*(Vec3D V)}}\label{Tensor2::operator*(Vec3D V)}\index[DL]{Tensor2!operator*(Vec3D V)}\\
Multiplication of a second order tensor by a vector.\\ \hspace*{10mm}$\hookrightarrow$ Vec3D

\begin{tcolorbox}[width=\textwidth,myArgs,tabularx={ll|R},title=Arguments of Tensor2::operator*]
Vec3D&V&Vec3D to use for the multiplication operation.
\end{tcolorbox}

This method defines the product of a second order tensor by a vector.
The result of this operation is also a vector defined by:
\begin{equation*}
\overrightarrow{y} = \A \cdot \overrightarrow{x}
\end{equation*}
where $\A$ is a second order tensor and $\overrightarrow{x}$ and $\overrightarrow{y}$ are two Vec3D.
%@END

%@DOC:Tensor2::getDeviator()
%Warning :
%This area is an automatic documentation generated from the DynELA source code.
%Do not change anything in this latex file between this position and the @END keyword.
\textcolor{purple}{\textbf{Tensor2::getDeviator(~)}}\label{Tensor2::getDeviator()}\index[DL]{Tensor2!getDeviator(~)}\\
Deviatoric part of a second order tensor.\\ \hspace*{10mm}$\hookrightarrow$ Tensor2

This method defines the deviatoric part of a second second order tensor.
The result of this operation is a second order tensor defined by the following equation:
\begin{equation*}
\Sig^d=\Sig-\frac{1}{3}\tr[\Sig].\Id
\end{equation*}
where $\Sig^d$ is the deviatoric part of the tensor, $\Sig$ is the tensor and $\Id$ is the unit tensor.
%@END

%@DOC:Tensor2::getSymetricPart()
%Warning :
%This area is an automatic documentation generated from the DynELA source code.
%Do not change anything in this latex file between this position and the @END keyword.
\textcolor{purple}{\textbf{Tensor2::getSymetricPart(~)}}\label{Tensor2::getSymetricPart()}\index[DL]{Tensor2!getSymetricPart(~)}\\
Symmetric part of a second order tensor.\\ \hspace*{10mm}$\hookrightarrow$ Tensor2

This method returns the symmetric part of a second second order tensor.
The result of this operation is a second second order tensor defined by:
\begin{equation*}
\B = \left[\begin{array}{ccc}
 A_{11} & \frac{A_{12} + A_{21}}{2} & \frac{A_{13} + A_{31}}{2}\\
 \frac{A_{12} + A_{21}}{2} & A_{22} & \frac {A_{23} + A_{32}}{2}\\
 \frac{A_{13} + A_{31}}{2} & \frac {A_{23} + A_{32}}{2} & A_{33}\end{array}
\right]
\end{equation*}
%@END

%@DOC:Tensor2::getSkewSymetricPart()
%Warning :
%This area is an automatic documentation generated from the DynELA source code.
%Do not change anything in this latex file between this position and the @END keyword.
\textcolor{purple}{\textbf{Tensor2::getSkewSymetricPart(~)}}\label{Tensor2::getSkewSymetricPart()}\index[DL]{Tensor2!getSkewSymetricPart(~)}\\
Skew-symmetric part of a second order tensor.\\ \hspace*{10mm}$\hookrightarrow$ Tensor2

This method returns the skew-symmetric part of a second second order tensor.
The result of this operation is a second second order tensor defined by:
\begin{equation*}
\B = \left[\begin{array}{ccc}
 A_{11} & \frac{A_{12} - A_{21}}{2} & \frac{A_{13} - A_{31}}{2}\\
 -\frac{A_{12} -  A_{21}}{2} & A_{22} & \frac {A_{23} - A_{32}}{2}\\
 -\frac{A_{13} - A_{31}}{2} & -\frac {A_{23} - A_{32}}{2} & A_{33}\end{array}
\right]
\end{equation*}
%@END

%@DOC:Tensor2::getDeterminant()
%Warning :
%This area is an automatic documentation generated from the DynELA source code.
%Do not change anything in this latex file between this position and the @END keyword.
\textcolor{purple}{\textbf{Tensor2::getDeterminant(~)}}\label{Tensor2::getDeterminant()}\index[DL]{Tensor2!getDeterminant(~)}\\
Determinant of a second order tensor.\\ \hspace*{10mm}$\hookrightarrow$ double

This method returns the determinant of a second second order tensor.
The result of this operation is a scalar value defined by:
\begin{equation*}
D = T_{11} T_{22} T_{33} + T_{21} T_{32} T_{13} + T_{31} T_{12} T_{23} - T_{31} T_{22} T_{13} - T_{11} T_{32} T_{23} - T_{21} T_{12} T_{33}
\end{equation*}
%@END

%@DOC:Tensor2::getInverse()
%Warning :
%This area is an automatic documentation generated from the DynELA source code.
%Do not change anything in this latex file between this position and the @END keyword.
\textcolor{purple}{\textbf{Tensor2::getInverse(~)}}\label{Tensor2::getInverse()}\index[DL]{Tensor2!getInverse(~)}\\
Inverse of a second order tensor.\\ \hspace*{10mm}$\hookrightarrow$ Tensor2

This method returns the inverse of a second second order tensor.
The result of this operation is a second order tensor defined by:
\begin{equation*}
D = T_{11} T_{22} T_{33} + T_{21} T_{32} T_{13} + T_{31} T_{12} T_{23} - T_{31} T_{22} T_{13} - T_{11} T_{32} T_{23} - T_{21} T_{12} T_{33}
\end{equation*}
\begin{equation*}
T^{-1} = \frac {1}{D} \left[\begin{array}{ccc}
  T_{22}T_{33}-T_{23}T_{32}&T_{13}T_{32}-T_{12}T_{33}&T_{12}T_{23}-T_{13}T_{22}\\
  T_{23}T_{31}-T_{21}T_{33}&T_{11}T_{33}-T_{13}T_{31}&T_{13}T_{21}-T_{11}T_{23}\\
  T_{21}T_{32}-T_{22}T_{31}&T_{12}T_{31}-T_{11}T_{32}&T_{11}T_{22}-T_{12}T_{21}
  \end{array}
  \right]
\end{equation*}
%@END

%@DOC:Tensor2::getNorm()
%Warning :
%This area is an automatic documentation generated from the DynELA source code.
%Do not change anything in this latex file between this position and the @END keyword.
\textcolor{purple}{\textbf{Tensor2::getNorm(~)}}\label{Tensor2::getNorm()}\index[DL]{Tensor2!getNorm(~)}\\
Norm of a second order tensor.\\ \hspace*{10mm}$\hookrightarrow$ double

This method returns the norm of a second order tensor defined by:\begin{equation*}
\left\Vert s \right\Vert  = \sqrt {s_{ij}:s_{ij}}
\end{equation*}
%@END

%@DOC:Tensor2::getJ2()
%Warning :
%This area is an automatic documentation generated from the DynELA source code.
%Do not change anything in this latex file between this position and the @END keyword.
\textcolor{purple}{\textbf{Tensor2::getJ2(~)}}\label{Tensor2::getJ2()}\index[DL]{Tensor2!getJ2(~)}\\
J2 norm of a second order tensor.\\ \hspace*{10mm}$\hookrightarrow$ double

This method returns the J2 norm of a second order tensor defined by:
\begin{equation*}
\sqrt {\frac{3}{2}} \left\Vert s \right\Vert  = \sqrt {\frac{3}{2} s_{ij}:s_{ij}}
\end{equation*}
%@END

\subsection{Advanced operations}

%@DOC:Tensor2::polarQL(SymTensor2 U, Tensor2 R)
%Warning :
%This area is an automatic documentation generated from the DynELA source code.
%Do not change anything in this latex file between this position and the @END keyword.
\textcolor{purple}{\textbf{Tensor2::polarQL(SymTensor2 U, Tensor2 R)}}\label{Tensor2::polarQL(SymTensor2 U, Tensor2 R)}\index[DL]{Tensor2!polarQL(SymTensor2 U, Tensor2 R)}\\
Polar decomposition of a second order tensor using the QL with implicit shifts algorithm.\\ \hspace*{10mm}$\hookrightarrow$ SymTensor2 and Tensor2

\begin{tcolorbox}[width=\textwidth,myArgs,tabularx={ll|R},title=Arguments of Tensor2::polarQL]
SymTensor2&U&Symmetric tensor $\U$\\
Tensor2&R&Rotation tensor $\R$
\end{tcolorbox}

This method computes the polar decomposition of a second order tensor $\F$ and returns the symmetric tensor $\R$ and the tensor $\U$ so that:
\begin{equation*}
\F = \R \cdot \U
\end{equation*}
%@END

%@DOC:Tensor2::polarQLLnU(SymTensor2 U, Tensor2 R)
%Warning :
%This area is an automatic documentation generated from the DynELA source code.
%Do not change anything in this latex file between this position and the @END keyword.
\textcolor{purple}{\textbf{Tensor2::polarQLLnU(SymTensor2 U, Tensor2 R)}}\label{Tensor2::polarQLLnU(SymTensor2 U, Tensor2 R)}\index[DL]{Tensor2!polarQLLnU(SymTensor2 U, Tensor2 R)}\\
Polar decomposition of a second order tensor using the QL with implicit shifts algorithm.\\ \hspace*{10mm}$\hookrightarrow$ SymTensor2 and Tensor2

\begin{tcolorbox}[width=\textwidth,myArgs,tabularx={ll|R},title=Arguments of Tensor2::polarQLLnU]
SymTensor2&U&Symmetric tensor $\log[\U]$\\
Tensor2&R&Rotation tensor $\R$
\end{tcolorbox}

This method computes the polar decomposition of a second order tensor $\F$ and returns the symmetric tensor $\R$ and the tensor $\log[\U]$ so that:
\begin{equation*}
\F = \R \cdot \U
\end{equation*}
\begin{equation*}
\log [\U] =\sum _{i=1}^{3}\log[\lambda_{i}](\overrightarrow{u}_{i}\otimes \overrightarrow{u}_{i})
\end{equation*}
%@END

%@DOC:Tensor2::polarCuppen(SymTensor2 U, Tensor2 R)
%Warning :
%This area is an automatic documentation generated from the DynELA source code.
%Do not change anything in this latex file between this position and the @END keyword.
\textcolor{purple}{\textbf{Tensor2::polarCuppen(SymTensor2 U, Tensor2 R)}}\label{Tensor2::polarCuppen(SymTensor2 U, Tensor2 R)}\index[DL]{Tensor2!polarCuppen(SymTensor2 U, Tensor2 R)}\\
Polar decomposition of a second order tensor using the Cuppen’s Divide and Conquer algorithm.\\ \hspace*{10mm}$\hookrightarrow$ SymTensor2 and Tensor2

\begin{tcolorbox}[width=\textwidth,myArgs,tabularx={ll|R},title=Arguments of Tensor2::polarCuppen]
SymTensor2&U&Symmetric tensor $\U$\\
Tensor2&R&Rotation tensor $\R$
\end{tcolorbox}

This method computes the polar decomposition of a second order tensor $\F$ and returns the symmetric tensor $\R$ and the tensor $\U$ so that:
\begin{equation*}
\F = \R \cdot \U
\end{equation*}
%@END

%@DOC:Tensor2::polarCuppenLnU(SymTensor2 U, Tensor2 R)
%Warning :
%This area is an automatic documentation generated from the DynELA source code.
%Do not change anything in this latex file between this position and the @END keyword.
\textcolor{purple}{\textbf{Tensor2::polarCuppenLnU(SymTensor2 U, Tensor2 R)}}\label{Tensor2::polarCuppenLnU(SymTensor2 U, Tensor2 R)}\index[DL]{Tensor2!polarCuppenLnU(SymTensor2 U, Tensor2 R)}\\
Polar decomposition of a second order tensor using the Cuppen’s Divide and Conquer algorithm.\\ \hspace*{10mm}$\hookrightarrow$ SymTensor2 and Tensor2

\begin{tcolorbox}[width=\textwidth,myArgs,tabularx={ll|R},title=Arguments of Tensor2::polarCuppenLnU]
SymTensor2&U&Symmetric tensor $\log[\U]$\\
Tensor2&R&Rotation tensor $\R$
\end{tcolorbox}

This method computes the polar decomposition of a second order tensor $\F$ and returns the symmetric tensor $\R$ and the tensor $\log[\U]$ so that:
\begin{equation*}
\F = \R \cdot \U
\end{equation*}
\begin{equation*}
\log [\U] =\sum _{i=1}^{3}\log[\lambda_{i}](\overrightarrow{u}_{i}\otimes \overrightarrow{u}_{i})
\end{equation*}
%@END

%@DOC:Tensor2::polarJacobi(SymTensor2 U, Tensor2 R)
%Warning :
%This area is an automatic documentation generated from the DynELA source code.
%Do not change anything in this latex file between this position and the @END keyword.
\textcolor{purple}{\textbf{Tensor2::polarJacobi(SymTensor2 U, Tensor2 R)}}\label{Tensor2::polarJacobi(SymTensor2 U, Tensor2 R)}\index[DL]{Tensor2!polarJacobi(SymTensor2 U, Tensor2 R)}\\
Polar decomposition of a second order tensor using the Jacobi algorithm.\\ \hspace*{10mm}$\hookrightarrow$ SymTensor2 and Tensor2

\begin{tcolorbox}[width=\textwidth,myArgs,tabularx={ll|R},title=Arguments of Tensor2::polarJacobi]
SymTensor2&U&Symmetric tensor $\U$\\
Tensor2&R&Rotation tensor $\R$
\end{tcolorbox}

This method computes the polar decomposition of a second order tensor $\F$ and returns the symmetric tensor $\R$ and the tensor $\U$ so that:
\begin{equation*}
\F = \R \cdot \U
\end{equation*}
%@END

%@DOC:Tensor2::polarJacobiLnU(SymTensor2 U, Tensor2 R)
%Warning :
%This area is an automatic documentation generated from the DynELA source code.
%Do not change anything in this latex file between this position and the @END keyword.
\textcolor{purple}{\textbf{Tensor2::polarJacobiLnU(SymTensor2 U, Tensor2 R)}}\label{Tensor2::polarJacobiLnU(SymTensor2 U, Tensor2 R)}\index[DL]{Tensor2!polarJacobiLnU(SymTensor2 U, Tensor2 R)}\\
Polar decomposition of a second order tensor using the Jacobi algorithm.\\ \hspace*{10mm}$\hookrightarrow$ SymTensor2 and Tensor2

\begin{tcolorbox}[width=\textwidth,myArgs,tabularx={ll|R},title=Arguments of Tensor2::polarJacobiLnU]
SymTensor2&U&Symmetric tensor $\log[\U]$\\
Tensor2&R&Rotation tensor $\R$
\end{tcolorbox}

This method computes the polar decomposition of a second order tensor $\F$ and returns the symmetric tensor $\R$ and the tensor $\log[\U]$ so that:
\begin{equation*}
\F = \R \cdot \U
\end{equation*}
\begin{equation*}
\log [\U] =\sum _{i=1}^{3}\log[\lambda_{i}](\overrightarrow{u}_{i}\otimes \overrightarrow{u}_{i})
\end{equation*}
%@END

%@DOC:Tensor2::polarLapack(SymTensor2 U, Tensor2 R)
%Warning :
%This area is an automatic documentation generated from the DynELA source code.
%Do not change anything in this latex file between this position and the @END keyword.
\textcolor{purple}{\textbf{Tensor2::polarLapack(SymTensor2 U, Tensor2 R)}}\label{Tensor2::polarLapack(SymTensor2 U, Tensor2 R)}\index[DL]{Tensor2!polarLapack(SymTensor2 U, Tensor2 R)}\\
Polar decomposition of a second order tensor using the Jacobi algorithm.\\ \hspace*{10mm}$\hookrightarrow$ SymTensor2 and Tensor2

\begin{tcolorbox}[width=\textwidth,myArgs,tabularx={ll|R},title=Arguments of Tensor2::polarLapack]
SymTensor2&U&Symmetric tensor $\U$\\
Tensor2&R&Rotation tensor $\R$
\end{tcolorbox}

This method computes the polar decomposition of a second order tensor $\F$ and returns the symmetric tensor $\R$ and the tensor $\U$ so that:
\begin{equation*}
\F = \R \cdot \U
\end{equation*}
It uses the LAPACKE\_dgeev function of the Lapack library which is far from efficient for a trivial 3x3 matrix. So this method is very slow.
%@END

%@DOC:Tensor2::polarLapackLnU(SymTensor2 U, Tensor2 R)
%Warning :
%This area is an automatic documentation generated from the DynELA source code.
%Do not change anything in this latex file between this position and the @END keyword.
\textcolor{purple}{\textbf{Tensor2::polarLapackLnU(SymTensor2 U, Tensor2 R)}}\label{Tensor2::polarLapackLnU(SymTensor2 U, Tensor2 R)}\index[DL]{Tensor2!polarLapackLnU(SymTensor2 U, Tensor2 R)}\\
Polar decomposition of a second order tensor using the Jacobi algorithm.\\ \hspace*{10mm}$\hookrightarrow$ SymTensor2 and Tensor2

\begin{tcolorbox}[width=\textwidth,myArgs,tabularx={ll|R},title=Arguments of Tensor2::polarLapackLnU]
SymTensor2&U&Symmetric tensor $\log[\U]$\\
Tensor2&R&Rotation tensor $\R$
\end{tcolorbox}

This method computes the polar decomposition of a second order tensor $\F$ and returns the symmetric tensor $\R$ and the tensor $\log[\U]$ so that:
\begin{equation*}
\F = \R \cdot \U
\end{equation*}
\begin{equation*}
\log [\U] =\sum _{i=1}^{3}\log[\lambda_{i}](\overrightarrow{u}_{i}\otimes \overrightarrow{u}_{i})
\end{equation*}
It uses the LAPACKE\_dgeev function of the Lapack library which is far from efficient for a trivial 3x3 matrix. So this method is very slow.
%@END

%@DOC:Tensor2::polar(SymTensor2 U, Tensor2 R)
%Warning :
%This area is an automatic documentation generated from the DynELA source code.
%Do not change anything in this latex file between this position and the @END keyword.
\textcolor{purple}{\textbf{Tensor2::polar(SymTensor2 U, Tensor2 R)}}\label{Tensor2::polar(SymTensor2 U, Tensor2 R)}\index[DL]{Tensor2!polar(SymTensor2 U, Tensor2 R)}\\
Polar decomposition of a second order tensor using the old \DynELA algorithm.\\ \hspace*{10mm}$\hookrightarrow$ SymTensor2 and Tensor2

\begin{tcolorbox}[width=\textwidth,myArgs,tabularx={ll|R},title=Arguments of Tensor2::polar]
SymTensor2&U&Symmetric tensor $\log[\U]$\\
Tensor2&R&Rotation tensor $\R$
\end{tcolorbox}

This method computes the polar decomposition of a second order tensor $\F$ and returns the symmetric tensor $\R$ and the tensor $\U$ so that:
\begin{equation*}
\F = \R \cdot \U
\end{equation*}
%@END

%@DOC:Tensor2::polarLnU(SymTensor2 U, Tensor2 R)
%Warning :
%This area is an automatic documentation generated from the DynELA source code.
%Do not change anything in this latex file between this position and the @END keyword.
\textcolor{purple}{\textbf{Tensor2::polarLnU(SymTensor2 U, Tensor2 R)}}\label{Tensor2::polarLnU(SymTensor2 U, Tensor2 R)}\index[DL]{Tensor2!polarLnU(SymTensor2 U, Tensor2 R)}\\
Polar decomposition of a second order tensor using the old \DynELA algorithm.\\ \hspace*{10mm}$\hookrightarrow$ SymTensor2 and Tensor2

\begin{tcolorbox}[width=\textwidth,myArgs,tabularx={ll|R},title=Arguments of Tensor2::polarLnU]
SymTensor2&U&Symmetric tensor $\U$\\
Tensor2&R&Rotation tensor $\R$
\end{tcolorbox}

This method computes the polar decomposition of a second order tensor $\F$ and returns the symmetric tensor $\R$ and the tensor $\log[\U]$ so that:
\begin{equation*}
\F = \R \cdot \U
\end{equation*}
\begin{equation*}
\log [\U] =\sum _{i=1}^{3}\log[\lambda_{i}](\overrightarrow{u}_{i}\otimes \overrightarrow{u}_{i})
\end{equation*}
%@END

\begin{tcolorbox}[width=0.95\textwidth,myTab,tabularx={l||C|C|C},title=Performance of the polar algorithms]%,boxrule=0.5pt]
 & $\U$ version& $\log[\U]$ & Precision\\
 & CPU (ns) & CPU (ns) & $\F-\R\cdot\U$\\\hline\hline
polarLapack & $2163$ & $2360$ & $1.4210\times10^{-14}$ \\\hline
polar & $397$ & $432$ & $3.5527\times10^{-15}$\\\hline
polarQL & $246$ & $292$ & $1.0658\times10^{-14}$\\\hline
polarJacobi & $368$ & $390$ & $5.3291\times10^{-15}$\\\hline
polarCuppen & $234$ & $243$ & $3.5527\times10^{-15}$
\end{tcolorbox}

\section{The Symmetric Tensor2 library}

%@DOC:SymTensor2::SymTensor2
%Warning :
%This area is an automatic documentation generated from the DynELA source code.
%Do not change anything in this latex file between this position and the @END keyword.
\textcolor{purple}{\textbf{SymTensor2::SymTensor2}}\label{SymTensor2::SymTensor2}\index[DL]{SymTensor2!SymTensor2}\\
Second order tensor class.

The SymTensor2 library is used to store symmetric second order tensors defined in the \DynELA. A symmetric second order tensor is a like a matrix with the following form:
\begin{equation*}
T=\left[\begin{array}{ccc}
  T_{11} & T_{12} & T_{13}\\
  T_{12} & T_{22} & T_{23}\\
  T_{13} & T_{23} & T_{33}
  \end{array}\right]
\end{equation*}
Concerning the internal storage of data, the SymTensor2 data is stored in a vector \_data of 6 components using the following storage scheme:
\begin{equation*}
T=\left[\begin{array}{ccc}
    T_{0} & T_{1} & T_{2}\\
    T_{1} & T_{3} & T_{4}\\
    T_{2} & T_{4} & T_{5}
    \end{array}\right]
\end{equation*}
%@END

\subsection{Constructor and destructor}

%@DOC:SymTensor2::SymTensor2()
%Warning :
%This area is an automatic documentation generated from the DynELA source code.
%Do not change anything in this latex file between this position and the @END keyword.
\textcolor{purple}{\textbf{SymTensor2::SymTensor2(~)}}\label{SymTensor2::SymTensor2()}\index[DL]{SymTensor2!SymTensor2(~)}\\
Default constructor of the SymTensor2 class.\\ \hspace*{10mm}$\hookrightarrow$ SymTensor2

All components are initialized to zero by default.
\begin{equation*}
\T=\left[\begin{array}{ccc}
0&0&0\\
0&0&0\\
0&0&0
\end{array}\right]
\end{equation*}
%@END

%@DOC:SymTensor2::SymTensor2(double,...)
%Warning :
%This area is an automatic documentation generated from the DynELA source code.
%Do not change anything in this latex file between this position and the @END keyword.
\textcolor{purple}{\textbf{SymTensor2::SymTensor2(double,...)}}\label{SymTensor2::SymTensor2(double,...)}\index[DL]{SymTensor2!SymTensor2(double,...)}\\
Constructor of the SymTensor2 class.\\ \hspace*{10mm}$\hookrightarrow$ SymTensor2

\begin{tcolorbox}[width=\textwidth,myArgs,tabularx={ll|R},title=Arguments of SymTensor2::SymTensor2]
double&t1&Component $t_{11}$ of the tensor.\\
double&t2&Component $t_{12}$ of the tensor.\\
double&t3&Component $t_{13}$ of the tensor.\\
double&t4&Component $t_{22}$ of the tensor.\\
double&t5&Component $t_{23}$ of the tensor.\\
double&t6&Component $t_{33}$ of the tensor.
\end{tcolorbox}

Constructor of a second order symmetric tensor with explicit initialization of the 6 components of the tensor.
%@END

%@DOC:SymTensor2::SymTensor2(SymTensor2 T)
%Warning :
%This area is an automatic documentation generated from the DynELA source code.
%Do not change anything in this latex file between this position and the @END keyword.
\textcolor{purple}{\textbf{SymTensor2::SymTensor2(SymTensor2 T)}}\label{SymTensor2::SymTensor2(SymTensor2 T)}\index[DL]{SymTensor2!SymTensor2(SymTensor2 T)}\\
Copy constructor of the SymTensor2 class.\\ \hspace*{10mm}$\hookrightarrow$ SymTensor2

\begin{tcolorbox}[width=\textwidth,myArgs,tabularx={ll|R},title=Arguments of SymTensor2::SymTensor2]
SymTensor2&T&Tensor to copy.
\end{tcolorbox}

%@END

%@DOC:SymTensor2::~SymTensor2()
%Warning :
%This area is an automatic documentation generated from the DynELA source code.
%Do not change anything in this latex file between this position and the @END keyword.
\textcolor{purple}{\textbf{SymTensor2::$\sim$SymTensor2(~)}}\label{SymTensor2::~SymTensor2()}\index[DL]{SymTensor2!$\sim$SymTensor2(~)}\\
Destructor of the SymTensor2 class.

%@END

\subsection{Basic operations}

%@DOC:SymTensor2::setToZero()
%Warning :
%This area is an automatic documentation generated from the DynELA source code.
%Do not change anything in this latex file between this position and the @END keyword.
\textcolor{purple}{\textbf{SymTensor2::setToZero(~)}}\label{SymTensor2::setToZero()}\index[DL]{SymTensor2!setToZero(~)}\\
Sets all components of the tensor to zero.

\hspace*{10mm}\textcolor{red}{\textbf{Warning :} This method modifies its own argument}

\begin{equation*}
\T=\left[\begin{array}{ccc}
0&0&0\\
0&0&0\\
0&0&0
\end{array}\right]
\end{equation*}
%@END

%@DOC:SymTensor2::setToUnity()
%Warning :
%This area is an automatic documentation generated from the DynELA source code.
%Do not change anything in this latex file between this position and the @END keyword.
\textcolor{purple}{\textbf{SymTensor2::setToUnity(~)}}\label{SymTensor2::setToUnity()}\index[DL]{SymTensor2!setToUnity(~)}\\
Unity tensor.

\hspace*{10mm}\textcolor{red}{\textbf{Warning :} This method modifies its own argument}

This method transforms the current tensor to a unity tensor.
\begin{equation*}
\T=\left[\begin{array}{ccc}
1&0&0\\
0&1&0\\
0&0&1
\end{array}\right]
\end{equation*}
%@END

%@DOC:SymTensor2::operator=(double val)
%Warning :
%This area is an automatic documentation generated from the DynELA source code.
%Do not change anything in this latex file between this position and the @END keyword.
\textcolor{purple}{\textbf{SymTensor2::operator=(double val)}}\label{SymTensor2::operator=(double val)}\index[DL]{SymTensor2!operator=(double val)}\\
Fill a second order tensor with a scalar value.\\ \hspace*{10mm}$\hookrightarrow$ SymTensor2

\begin{tcolorbox}[width=\textwidth,myArgs,tabularx={ll|R},title=Arguments of SymTensor2::operator=]
double&val&Value to use for the operation.
\end{tcolorbox}

This method is a surdefinition of the = operator for the symmetric second order tensor class.
\begin{equation*}
\T=\left[\begin{array}{ccc}
m&m&m\\
m&m&m\\
m&m&m
\end{array}\right]
\end{equation*}
%@END

%@DOC:SymTensor2::rowSum()
%Warning :
%This area is an automatic documentation generated from the DynELA source code.
%Do not change anything in this latex file between this position and the @END keyword.
\textcolor{purple}{\textbf{SymTensor2::rowSum(~)}}\label{SymTensor2::rowSum()}\index[DL]{SymTensor2!rowSum(~)}\\
Sum of the rows of a second order tensor.\\ \hspace*{10mm}$\hookrightarrow$ Vec3D

This method returns a vector by computing the sum of the components on all rows of a second second order tensor.
The result of this operation is a vector defined by:
\begin{equation*}
v_{i}=\sum_{j=1}^{3} T_{ji}
\end{equation*}
%@END

%@DOC:SymTensor2::columnSum()
%Warning :
%This area is an automatic documentation generated from the DynELA source code.
%Do not change anything in this latex file between this position and the @END keyword.
\textcolor{purple}{\textbf{SymTensor2::columnSum(~)}}\label{SymTensor2::columnSum()}\index[DL]{SymTensor2!columnSum(~)}\\
Sum of the columns of a second order tensor.\\ \hspace*{10mm}$\hookrightarrow$ Vec3D

This method returns a vector by computing the sum of the components on all columns of a second second order tensor.
The result of this operation is a vector defined by:
\begin{equation*}
v_{i}=\sum_{j=1}^{3}T_{ij}
\end{equation*}
%@END

%@DOC:SymTensor2::getRow(short row)
%Warning :
%This area is an automatic documentation generated from the DynELA source code.
%Do not change anything in this latex file between this position and the @END keyword.
\textcolor{purple}{\textbf{SymTensor2::getRow(short row)}}\label{SymTensor2::getRow(short row)}\index[DL]{SymTensor2!getRow(short row)}\\
Extraction of a row from a second order tensor.\\ \hspace*{10mm}$\hookrightarrow$ Vec3D

\begin{tcolorbox}[width=\textwidth,myArgs,tabularx={ll|R},title=Arguments of SymTensor2::getRow]
short&row&Row to extract
\end{tcolorbox}

This method returns a vector as part of a second second order tensor.
The result of this operation with the argument j is a vector defined by:
\begin{equation*}
v_{i} = T_{ij}
\end{equation*}
%@END

%@DOC:SymTensor2::getColumn(short col)
%Warning :
%This area is an automatic documentation generated from the DynELA source code.
%Do not change anything in this latex file between this position and the @END keyword.
\textcolor{purple}{\textbf{SymTensor2::getColumn(short col)}}\label{SymTensor2::getColumn(short col)}\index[DL]{SymTensor2!getColumn(short col)}\\
Extraction of a column from a second order tensor.\\ \hspace*{10mm}$\hookrightarrow$ Vec3D

\begin{tcolorbox}[width=\textwidth,myArgs,tabularx={ll|R},title=Arguments of SymTensor2::getColumn]
short&col&Column to extract
\end{tcolorbox}

This method returns a vector as part of a second second order tensor.
The result of this operation with the argument j is a vector defined by:
\begin{equation*}
v_{i} = T_{ji}
\end{equation*}
%@END

%@DOC:SymTensor2::minValue()
%Warning :
%This area is an automatic documentation generated from the DynELA source code.
%Do not change anything in this latex file between this position and the @END keyword.
\textcolor{purple}{\textbf{SymTensor2::minValue(~)}}\label{SymTensor2::minValue()}\index[DL]{SymTensor2!minValue(~)}\\
Minimum component in a second order tensor.\\ \hspace*{10mm}$\hookrightarrow$ double

This method returns the minimum component in a second second order tensor.
%@END

%@DOC:SymTensor2::minAbsoluteValue()
%Warning :
%This area is an automatic documentation generated from the DynELA source code.
%Do not change anything in this latex file between this position and the @END keyword.
\textcolor{purple}{\textbf{SymTensor2::minAbsoluteValue(~)}}\label{SymTensor2::minAbsoluteValue()}\index[DL]{SymTensor2!minAbsoluteValue(~)}\\
Minimum absolute component in a second order tensor.\\ \hspace*{10mm}$\hookrightarrow$ double

This method returns the minimum absolute component in a second second order tensor.
%@END

%@DOC:SymTensor2::maxValue()
%Warning :
%This area is an automatic documentation generated from the DynELA source code.
%Do not change anything in this latex file between this position and the @END keyword.
\textcolor{purple}{\textbf{SymTensor2::maxValue(~)}}\label{SymTensor2::maxValue()}\index[DL]{SymTensor2!maxValue(~)}\\
Maximum component in a second order tensor.\\ \hspace*{10mm}$\hookrightarrow$ double

This method returns the maximum component in a second second order tensor.
%@END

%@DOC:SymTensor2::maxAbsoluteValue()
%Warning :
%This area is an automatic documentation generated from the DynELA source code.
%Do not change anything in this latex file between this position and the @END keyword.
\textcolor{purple}{\textbf{SymTensor2::maxAbsoluteValue(~)}}\label{SymTensor2::maxAbsoluteValue()}\index[DL]{SymTensor2!maxAbsoluteValue(~)}\\
Maximum absolute component in a second order tensor.\\ \hspace*{10mm}$\hookrightarrow$ double

This method returns the maximum absolute component in a second second order tensor.
%@END

%@DOC:SymTensor2::getTrace()
%Warning :
%This area is an automatic documentation generated from the DynELA source code.
%Do not change anything in this latex file between this position and the @END keyword.
\textcolor{purple}{\textbf{SymTensor2::getTrace(~)}}\label{SymTensor2::getTrace()}\index[DL]{SymTensor2!getTrace(~)}\\
Returns the trace of a symmetric second order tensor.\\ \hspace*{10mm}$\hookrightarrow$ double

  This method returns the trace of a symmetric second order tensor, i.e. the sum of all the terms of the diagonal:
\begin{equation*}
v = tr[\T] = T_{11}+T_{22}+T_{33}
\end{equation*}
%@END

%@DOC:SymTensor2::getThirdTrace()
%Warning :
%This area is an automatic documentation generated from the DynELA source code.
%Do not change anything in this latex file between this position and the @END keyword.
\textcolor{purple}{\textbf{SymTensor2::getThirdTrace(~)}}\label{SymTensor2::getThirdTrace()}\index[DL]{SymTensor2!getThirdTrace(~)}\\
Returns the average value of the trace of a symmetric second order tensor.\\ \hspace*{10mm}$\hookrightarrow$ double

This method returns average value of the trace of a symmetric second order tensor, i.e. the sum of all the terms of the diagonal divided by 3:
\begin{equation*}
v = \frac{1}{3} tr[\T] =  \frac{1}{3} \left( T_{11}+T_{22}+T_{33} \right)
\end{equation*}
%@END

\subsection{Specific operations}

%@DOC:SymTensor2::singleProduct(SymTensor2 T)
%Warning :
%This area is an automatic documentation generated from the DynELA source code.
%Do not change anything in this latex file between this position and the @END keyword.
\textcolor{purple}{\textbf{SymTensor2::singleProduct(SymTensor2 T)}}\label{SymTensor2::singleProduct(SymTensor2 T)}\index[DL]{SymTensor2!singleProduct(SymTensor2 T)}\\
Contracted product of two symmetric second order tensors.\\ \hspace*{10mm}$\hookrightarrow$ Tensor2

\begin{tcolorbox}[width=\textwidth,myArgs,tabularx={ll|R},title=Arguments of SymTensor2::singleProduct]
SymTensor2&T&Second Tensor for the multiplication operation.
\end{tcolorbox}

This method defines a single contracted product of two symmetric second order tensors.
The result of this operation is also a second order tensor defined by:
\begin{equation*}
\T=A \cdot \B=\left[\begin{array}{ccc}
A_{11} B_{11} + A_{12} B_{12} + A_{13} B_{13} & A_{11} B_{12} + A_{12} B_{22} + A_{13} B_{23} & A_{11} B_{13} + A_{12} B_{23} + A_{13} B_{33} \\
A_{12} B_{11} + A_{22} B_{12} + A_{23} B_{13} & A_{12} B_{12} + A_{22} B_{22} + A_{23} B_{23} & A_{12} B_{13} + A_{22} B_{23} + A_{23} B_{33} \\
A_{13} B_{11} + A_{23} B_{12} + A_{33} B_{13} & A_{13} B_{12} + A_{23} B_{22} + A_{33} B_{23} & A_{13} B_{13} + A_{23} B_{23} + A_{33} B_{33}
\end{array}\right]
\end{equation*}
where $\A$ and $\B$ are two symmetric second order tensors, the result is a non symmetric second order tensor.
%@END

%@DOC:SymTensor2::singleProduct()
%Warning :
%This area is an automatic documentation generated from the DynELA source code.
%Do not change anything in this latex file between this position and the @END keyword.
\textcolor{purple}{\textbf{SymTensor2::singleProduct(~)}}\label{SymTensor2::singleProduct()}\index[DL]{SymTensor2!singleProduct(~)}\\
Contracted product of a symmetric second order tensor by itself.\\ \hspace*{10mm}$\hookrightarrow$ SymTensor2

This method defines a single contracted product of of a symmetric second order tensor by itself.
The result of this operation is also a symmetric second order tensor defined by:
\begin{equation*}
\T = \A \cdot \A
\end{equation*}
where $\A$ is a two symmetric second order tensor.
%@END

%@DOC:SymTensor2::operator*(SymTensor2 T)
%Warning :
%This area is an automatic documentation generated from the DynELA source code.
%Do not change anything in this latex file between this position and the @END keyword.
\textcolor{purple}{\textbf{SymTensor2::operator*(SymTensor2 T)}}\label{SymTensor2::operator*(SymTensor2 T)}\index[DL]{SymTensor2!operator*(SymTensor2 T)}\\
Multiplication of 2 second order tensors.\\ \hspace*{10mm}$\hookrightarrow$ Tensor2

\begin{tcolorbox}[width=\textwidth,myArgs,tabularx={ll|R},title=Arguments of SymTensor2::operator*]
Tensor2&T&Second Tensor for the multiplication operation.
\end{tcolorbox}

This method defines a single contracted product of two symmetric second order tensors.
The result of this operation is also a second order tensor defined by:
\begin{equation*}
\T = \A \cdot \B
\end{equation*}
where $\A$ and $\B$ are two symmetric second order tensors.
%@END

%@DOC:SymTensor2::operator*(Tensor2 T)
%Warning :
%This area is an automatic documentation generated from the DynELA source code.
%Do not change anything in this latex file between this position and the @END keyword.
\textcolor{purple}{\textbf{SymTensor2::operator*(Tensor2 T)}}\label{SymTensor2::operator*(Tensor2 T)}\index[DL]{SymTensor2!operator*(Tensor2 T)}\\
Multiplication of 2 second order tensors.\\ \hspace*{10mm}$\hookrightarrow$ Tensor2

\begin{tcolorbox}[width=\textwidth,myArgs,tabularx={ll|R},title=Arguments of SymTensor2::operator*]
Tensor2&T&Second Tensor for the multiplication operation.
\end{tcolorbox}

This method defines a single contracted product of two symmetric second order tensors.
The result of this operation is also a second order tensor defined by:
\begin{equation*}
\T = \A \cdot \B
\end{equation*}
where $\B$ is a second order tensor and $\A$ is a symmetric second order tensor.
%@END

%@DOC:SymTensor2::doubleProduct(SymTensor2 T)
%Warning :
%This area is an automatic documentation generated from the DynELA source code.
%Do not change anything in this latex file between this position and the @END keyword.
\textcolor{purple}{\textbf{SymTensor2::doubleProduct(SymTensor2 T)}}\label{SymTensor2::doubleProduct(SymTensor2 T)}\index[DL]{SymTensor2!doubleProduct(SymTensor2 T)}\\
Double contracted product of 2 second order tensors.\\ \hspace*{10mm}$\hookrightarrow$ double

\begin{tcolorbox}[width=\textwidth,myArgs,tabularx={ll|R},title=Arguments of SymTensor2::doubleProduct]
SymTensor2&T&Second Tensor for the multiplication operation.
\end{tcolorbox}

This method defines a double contracted product of two symmetric second order tensors.
The result of this operation is a scalar defined by:
\begin{equation*}
s = \A : \B = \sum_{i=1}^{3} \sum_{j=1}^{3} A_{ij}\times B_{ij}
\end{equation*}
where $\A$ and $\B$ are two symmetric second order tensors.
%@END

%@DOC:SymTensor2::doubleProduct()
%Warning :
%This area is an automatic documentation generated from the DynELA source code.
%Do not change anything in this latex file between this position and the @END keyword.
\textcolor{purple}{\textbf{SymTensor2::doubleProduct(~)}}\label{SymTensor2::doubleProduct()}\index[DL]{SymTensor2!doubleProduct(~)}\\
Double contracted product of a second order tensor by itself.\\ \hspace*{10mm}$\hookrightarrow$ double

This method defines a double contracted product of a second order tensor by itself.
The result of this operation is a scalar defined by:
\begin{equation*}
s = \A : \A = \sum_{i=1}^{3} \sum_{j=1}^{3} A_{ij}\times A_{ij}
\end{equation*}
where $\A$ is a second order tensor.
%@END

%@DOC:SymTensor2::operator*(Vec3D V)
%Warning :
%This area is an automatic documentation generated from the DynELA source code.
%Do not change anything in this latex file between this position and the @END keyword.
\textcolor{purple}{\textbf{SymTensor2::operator*(Vec3D V)}}\label{SymTensor2::operator*(Vec3D V)}\index[DL]{SymTensor2!operator*(Vec3D V)}\\
Multiplication of a symmetric second order tensor by a vector.\\ \hspace*{10mm}$\hookrightarrow$ Vec3D

\begin{tcolorbox}[width=\textwidth,myArgs,tabularx={ll|R},title=Arguments of SymTensor2::operator*]
Vec3D&V&Vec3D to use for the multiplication operation.
\end{tcolorbox}

This method defines the product of a symmetric second order tensor by a vector.
The result of this operation is also a vector defined by:
\begin{equation*}
\overrightarrow{y} = \A \cdot \overrightarrow{x}
\end{equation*}
where $\A$ is a symmetric second order tensor and $\overrightarrow{x}$ and $\overrightarrow{y}$ are two Vec3D.
%@END

%@DOC:SymTensor2::getDeviator()
%Warning :
%This area is an automatic documentation generated from the DynELA source code.
%Do not change anything in this latex file between this position and the @END keyword.
\textcolor{purple}{\textbf{SymTensor2::getDeviator(~)}}\label{SymTensor2::getDeviator()}\index[DL]{SymTensor2!getDeviator(~)}\\
Deviatoric part of a second order tensor.\\ \hspace*{10mm}$\hookrightarrow$ SymTensor2

This method defines the deviatoric part of a second second order tensor.
The result of this operation is a second order tensor defined by the following equation:
\begin{equation*}
\Sig^d=\Sig-\frac{1}{3}\tr[\Sig].\Id
\end{equation*}
where $\Sig^d$ is the deviatoric part of the tensor, $\Sig$ is the tensor and $\Id$ is the unit tensor.
%@END

%@DOC:SymTensor2::getDeterminant()
%Warning :
%This area is an automatic documentation generated from the DynELA source code.
%Do not change anything in this latex file between this position and the @END keyword.
\textcolor{purple}{\textbf{SymTensor2::getDeterminant(~)}}\label{SymTensor2::getDeterminant()}\index[DL]{SymTensor2!getDeterminant(~)}\\
Determinant of a symmetric second order tensor.\\ \hspace*{10mm}$\hookrightarrow$ double

This method returns the determinant of a symmetric second second order tensor.
The result of this operation is a scalar value defined by:
\begin{equation*}
D = T_{11} T_{22} T_{33} + 2 T_{12} T_{23} T_{13} - T_{22} T_{13}^2 - T_{11} T_{23}^2 - T_{33} T_{12}^2
\end{equation*}
%@END

%@DOC:SymTensor2::getInverse()
%Warning :
%This area is an automatic documentation generated from the DynELA source code.
%Do not change anything in this latex file between this position and the @END keyword.
\textcolor{purple}{\textbf{SymTensor2::getInverse(~)}}\label{SymTensor2::getInverse()}\index[DL]{SymTensor2!getInverse(~)}\\
Inverse of a second order tensor.\\ \hspace*{10mm}$\hookrightarrow$ SymTensor2

This method returns the inverse of a second second order tensor.
The result of this operation is a second order tensor defined by:
\begin{equation*}
D = T_{11} T_{22} T_{33} + 2 T_{12} T_{23} T_{13} - T_{22} T_{13}^2 - T_{11} T_{23}^2 - T_{33} T_{12}^2
\end{equation*}
\begin{equation*}
T^{-1} = \frac {1}{D} \left[\begin{array}{ccc}
  T_{22}T_{33}-T_{23}^2&T_{13}T_{23}-T_{12}T_{33}&T_{12}T_{23}-T_{13}T_{22}\\
  T_{13}T_{23}-T_{12}T_{33}&T_{11}T_{33}-T_{13}^2&T_{12}T_{13}-T_{11}T_{23}\\
  T_{12}T_{23}-T_{13}T_{22}&T_{12}T_{13}-T_{11}T_{23}&T_{11}T_{22}-T_{12}^2
  \end{array}
  \right]
\end{equation*}
%@END

%@DOC:SymTensor2::getNorm()
%Warning :
%This area is an automatic documentation generated from the DynELA source code.
%Do not change anything in this latex file between this position and the @END keyword.
\textcolor{purple}{\textbf{SymTensor2::getNorm(~)}}\label{SymTensor2::getNorm()}\index[DL]{SymTensor2!getNorm(~)}\\
Norm of a symmetric second order tensor.\\ \hspace*{10mm}$\hookrightarrow$ double

This method returns the norm of a symmetric second order tensor defined by:
\begin{equation*}
\left\Vert s \right\Vert  = \sqrt {s_{ij}:s_{ij}}
\end{equation*}
%@END

%@DOC:SymTensor2::getJ2()
%Warning :
%This area is an automatic documentation generated from the DynELA source code.
%Do not change anything in this latex file between this position and the @END keyword.
\textcolor{purple}{\textbf{SymTensor2::getJ2(~)}}\label{SymTensor2::getJ2()}\index[DL]{SymTensor2!getJ2(~)}\\
J2 norm of a symmetric second order tensor.\\ \hspace*{10mm}$\hookrightarrow$ double

This method returns the J2 norm of a symmetric second order tensor defined by:
\begin{equation*}
\sqrt {\frac{3}{2}} \left\Vert s \right\Vert  = \sqrt {\frac{3}{2} s_{ij}:s_{ij}}
\end{equation*}
%@END

%@DOC:SymTensor2::getMisesEquivalent()
%Warning :
%This area is an automatic documentation generated from the DynELA source code.
%Do not change anything in this latex file between this position and the @END keyword.
\textcolor{purple}{\textbf{SymTensor2::getMisesEquivalent(~)}}\label{SymTensor2::getMisesEquivalent()}\index[DL]{SymTensor2!getMisesEquivalent(~)}\\
Returns the von Mises stress of a symmetric second order tensor.\\ \hspace*{10mm}$\hookrightarrow$ double

This method returns the von Mises stress of a symmetric second order tensor defined by:
\begin{equation*}
\overline{\sigma} = \frac {1}{\sqrt{2}}\sqrt{(s_{11}-s_{22})^2+(s_{22}-s_{33})^2+(s_{33}-s_{11})^2+6(s_{12}^2+s_{23}^2+s_{31}^2)}
\end{equation*}
%@END

\subsection{Advanced operations}

%@DOC:SymTensor2::productByRxRT(Tensor2 R)
%Warning :
%This area is an automatic documentation generated from the DynELA source code.
%Do not change anything in this latex file between this position and the @END keyword.
\textcolor{purple}{\textbf{SymTensor2::productByRxRT(Tensor2 R)}}\label{SymTensor2::productByRxRT(Tensor2 R)}\index[DL]{SymTensor2!productByRxRT(Tensor2 R)}\\
Special combination for a multiplication of 3 second order tensors.\\ \hspace*{10mm}$\hookrightarrow$ Tensor2

\begin{tcolorbox}[width=\textwidth,myArgs,tabularx={ll|R},title=Arguments of SymTensor2::productByRxRT]
Tensor2&R&Second Tensor for the multiplication operation.
\end{tcolorbox}

This method defines the product of a symmetric tensor by two rotations defined by the following equation:
\begin{equation*}
\R = \Q \cdot \A \cdot \Q^T =\left[\begin{array}{ccc}
R_0&R_1&R_2\\
&R_3&R_4\\
&&R_5
\end{array}\right]
\end{equation*}
with:
\begin{align*}
R_0&=A_0 Q_0^2 + 2 (A_1 Q_1 + A_2 Q_2)Q_0 + A_3 Q_1^2 + 2 A_4 Q_1 Q_2 + A_5 Q_2^2\\
R_1&=(A_0 Q_3 + A_1 Q_4 + A_2 Q_5)Q_0 + (A_1 Q_3 + A_3 Q_4 + A_4 Q_5)Q_1 + (A_2 Q_3 + A_4 Q_4 + A_5 Q_5)Q_2\\
R_2&= (A_0 Q_6 + A_1 Q_7 + A_2 Q_8)Q_0 + (A_1 Q_6 + A_3 Q_7 + A_4 Q_8)Q_1 + (A_2 Q_6 + A_4 Q_7 + A_5 Q_8)Q_2\\
R_3&=A_0 Q_3^2 + 2 (A_1 Q_4 + A_2 Q_5)Q_3+ A_3 Q_4^2 + 2 A_4 Q_4 Q_5 + A_5 Q_5^2\\
R_4&= (A_0 Q_6 + A_1 Q_7 + A_2 Q_8)Q_3 + (A_1 Q_6 + A_3 Q_7 + A_4 Q_8)Q_4 + (A_2 Q_6 + A_4 Q_7 + A_5 Q_8)Q_5\\
R_5&=A_0 Q_6^2 + 2 (A_1 Q_7 + A_2 Q_8)Q_6+ A_3 Q_7^2 + 2 A_4 Q_7 Q_8 + A_5 Q_8^2
\end{align*}
where $\A$ is a symmetric second order tensor and $\Q$ an orthogonal tensor.
%@END

%@DOC:SymTensor2::productByRTxR(Tensor2 R)
%Warning :
%This area is an automatic documentation generated from the DynELA source code.
%Do not change anything in this latex file between this position and the @END keyword.
\textcolor{purple}{\textbf{SymTensor2::productByRTxR(Tensor2 R)}}\label{SymTensor2::productByRTxR(Tensor2 R)}\index[DL]{SymTensor2!productByRTxR(Tensor2 R)}\\
Special combination for a multiplication of 3 second order tensors.\\ \hspace*{10mm}$\hookrightarrow$ Tensor2

\begin{tcolorbox}[width=\textwidth,myArgs,tabularx={ll|R},title=Arguments of SymTensor2::productByRTxR]
Tensor2&R&Second Tensor for the multiplication operation.
\end{tcolorbox}

This method defines the product of a symmetric tensor by two rotations defined by the following equation:
\begin{equation*}
\R = \Q^T \cdot \A \cdot \Q =\left[\begin{array}{ccc}
R_0&R_1&R_2\\
&R_3&R_4\\
&&R_5
\end{array}\right]
\end{equation*}
with:
\begin{align*}
R_0 &= A_0 Q_0^2 + 2 (A_1 Q_3 + A_2 Q_6) Q_0 + A_3 Q_3^2 + 2 A_4 Q_3 Q_6 + A_5 Q_6^2 \\
R_1 &= A_1 Q_1 Q_3 +A_2 Q_1 Q_6 + (A_0 Q_1 + A_1 Q_4 + A_2 Q_7) Q_0+ A_3 Q_3 Q_4 + A_4 Q_4 Q_6 + A_4 Q_3 Q_7 + A_5 Q_6 Q_7\\
R_2 &= A_1 Q_2 Q_3 + A_2 Q_2 Q_6 + Q_0 (A_0 Q_2 + A_1 Q_5 + A_2 Q_8) + A_3 Q_3 Q_5 + A_4 Q_5 Q_6 + A_4 Q_3 Q_8 + A_5 Q_6 Q_8\\
R_3 &= A_0 Q_1^2 + 2 (A_1 Q_4 + A_2 Q_7)Q_1 + A_3 Q_4^2 + 2 A_4 Q_4 Q_7 + A_5 Q_7^2\\
R_4 &= A_1 Q_2 Q_4 + A_2 Q_2 Q_7 + (A_0 Q_2 + A_1 Q_5 + A_2 Q_8)Q_1 + A_3 Q_4 Q_5 + A_4 Q_5 Q_7 + A_4 Q_4 Q_8 + A_5 Q_7 Q_8\\
R_5 &= A_0 Q_2^2 + 2 (A_1 Q_5 + A_2 Q_8) Q_2 + A_3 Q_5^2 + 2 A_4 Q_5 Q_8 + A_5 Q_8^2
\end{align*}
where $\A$ is a symmetric second order tensor and $\Q$ an orthogonal tensor.
%@END

%@DOC:SymTensor2::polarQL(SymTensor2 U, Tensor2 R)
%Warning :
%This area is an automatic documentation generated from the DynELA source code.
%Do not change anything in this latex file between this position and the @END keyword.
\textcolor{purple}{\textbf{SymTensor2::polarQL(SymTensor2 U, Tensor2 R)}}\label{SymTensor2::polarQL(SymTensor2 U, Tensor2 R)}\index[DL]{SymTensor2!polarQL(SymTensor2 U, Tensor2 R)}\\
Polar decomposition of a second order tensor using the QL with implicit shifts algorithm.\\ \hspace*{10mm}$\hookrightarrow$ SymTensor2 and Tensor2

\begin{tcolorbox}[width=\textwidth,myArgs,tabularx={ll|R},title=Arguments of SymTensor2::polarQL]
SymTensor2&U&Symmetric tensor $\U$\\
Tensor2&R&Rotation tensor $\R$
\end{tcolorbox}

This method computes the polar decomposition of a second order tensor $\F$ and returns the symmetric tensor $\R$ and the tensor $\U$ so that:
\begin{equation*}
\F = \R \cdot \U
\end{equation*}
%@END

%@DOC:SymTensor2::polarQLLnU(SymTensor2 U, Tensor2 R)
%Warning :
%This area is an automatic documentation generated from the DynELA source code.
%Do not change anything in this latex file between this position and the @END keyword.
\textcolor{purple}{\textbf{SymTensor2::polarQLLnU(SymTensor2 U, Tensor2 R)}}\label{SymTensor2::polarQLLnU(SymTensor2 U, Tensor2 R)}\index[DL]{SymTensor2!polarQLLnU(SymTensor2 U, Tensor2 R)}\\
Polar decomposition of a second order tensor using the QL with implicit shifts algorithm.\\ \hspace*{10mm}$\hookrightarrow$ SymTensor2 and Tensor2

\begin{tcolorbox}[width=\textwidth,myArgs,tabularx={ll|R},title=Arguments of SymTensor2::polarQLLnU]
SymTensor2&U&Symmetric tensor $\log[\U]$\\
Tensor2&R&Rotation tensor $\R$
\end{tcolorbox}

This method computes the polar decomposition of a second order tensor $\F$ and returns the symmetric tensor $\R$ and the tensor $\log[\U]$ so that:
\begin{equation*}
\F = \R \cdot \U
\end{equation*}
\begin{equation*}
\log [\U] =\sum _{i=1}^{3}\log[\lambda_{i}](\overrightarrow{u}_{i}\otimes \overrightarrow{u}_{i})
\end{equation*}
%@END

%@DOC:SymTensor2::polarCuppen(SymTensor2 U, Tensor2 R)
%Warning :
%This area is an automatic documentation generated from the DynELA source code.
%Do not change anything in this latex file between this position and the @END keyword.
\textcolor{purple}{\textbf{SymTensor2::polarCuppen(SymTensor2 U, Tensor2 R)}}\label{SymTensor2::polarCuppen(SymTensor2 U, Tensor2 R)}\index[DL]{SymTensor2!polarCuppen(SymTensor2 U, Tensor2 R)}\\
Polar decomposition of a second order tensor using the Cuppen’s Divide and Conquer algorithm.\\ \hspace*{10mm}$\hookrightarrow$ SymTensor2 and Tensor2

\begin{tcolorbox}[width=\textwidth,myArgs,tabularx={ll|R},title=Arguments of SymTensor2::polarCuppen]
SymTensor2&U&Symmetric tensor $\log[\U]$\\
Tensor2&R&Rotation tensor $\R$
\end{tcolorbox}

This method computes the polar decomposition of a second order tensor $\F$ and returns the symmetric tensor $\R$ and the tensor $\U$ so that:
\begin{equation*}
\F = \R \cdot \U
\end{equation*}
%@END

%@DOC:SymTensor2::polarCuppenLnU(SymTensor2 U, Tensor2 R)
%Warning :
%This area is an automatic documentation generated from the DynELA source code.
%Do not change anything in this latex file between this position and the @END keyword.
\textcolor{purple}{\textbf{SymTensor2::polarCuppenLnU(SymTensor2 U, Tensor2 R)}}\label{SymTensor2::polarCuppenLnU(SymTensor2 U, Tensor2 R)}\index[DL]{SymTensor2!polarCuppenLnU(SymTensor2 U, Tensor2 R)}\\
Polar decomposition of a second order tensor using the Cuppen’s Divide and Conquer algorithm.\\ \hspace*{10mm}$\hookrightarrow$ SymTensor2 and Tensor2

\begin{tcolorbox}[width=\textwidth,myArgs,tabularx={ll|R},title=Arguments of SymTensor2::polarCuppenLnU]
SymTensor2&U&Symmetric tensor $\log[\U]$\\
Tensor2&R&Rotation tensor $\R$
\end{tcolorbox}

This method computes the polar decomposition of a second order tensor $\F$ and returns the symmetric tensor $\R$ and the tensor $\log[\U]$ so that:
\begin{equation*}
\F = \R \cdot \U
\end{equation*}
\begin{equation*}
\log [\U] =\sum _{i=1}^{3}\log[\lambda_{i}](\overrightarrow{u}_{i}\otimes \overrightarrow{u}_{i})
\end{equation*}
%@END

%@DOC:SymTensor2::polarJacobi(SymTensor2 U, Tensor2 R)
%Warning :
%This area is an automatic documentation generated from the DynELA source code.
%Do not change anything in this latex file between this position and the @END keyword.
\textcolor{purple}{\textbf{SymTensor2::polarJacobi(SymTensor2 U, Tensor2 R)}}\label{SymTensor2::polarJacobi(SymTensor2 U, Tensor2 R)}\index[DL]{SymTensor2!polarJacobi(SymTensor2 U, Tensor2 R)}\\
Polar decomposition of a second order tensor using the Jacobi algorithm.\\ \hspace*{10mm}$\hookrightarrow$ SymTensor2 and Tensor2

\begin{tcolorbox}[width=\textwidth,myArgs,tabularx={ll|R},title=Arguments of SymTensor2::polarJacobi]
SymTensor2&U&Symmetric tensor $\log[\U]$\\
Tensor2&R&Rotation tensor $\R$
\end{tcolorbox}

This method computes the polar decomposition of a second order tensor $\F$ and returns the symmetric tensor $\R$ and the tensor $\U$ so that:
\begin{equation*}
\F = \R \cdot \U
\end{equation*}
%@END

%@DOC:SymTensor2::polarJacobiLnU(SymTensor2 U, Tensor2 R)
%Warning :
%This area is an automatic documentation generated from the DynELA source code.
%Do not change anything in this latex file between this position and the @END keyword.
\textcolor{purple}{\textbf{SymTensor2::polarJacobiLnU(SymTensor2 U, Tensor2 R)}}\label{SymTensor2::polarJacobiLnU(SymTensor2 U, Tensor2 R)}\index[DL]{SymTensor2!polarJacobiLnU(SymTensor2 U, Tensor2 R)}\\
Polar decomposition of a second order tensor using the Jacobi algorithm.\\ \hspace*{10mm}$\hookrightarrow$ SymTensor2 and Tensor2

\begin{tcolorbox}[width=\textwidth,myArgs,tabularx={ll|R},title=Arguments of SymTensor2::polarJacobiLnU]
SymTensor2&U&Symmetric tensor $\log[\U]$\\
Tensor2&R&Rotation tensor $\R$
\end{tcolorbox}

This method computes the polar decomposition of a second order tensor $\F$ and returns the symmetric tensor $\R$ and the tensor $\log[\U]$ so that:
\begin{equation*}
\F = \R \cdot \U
\end{equation*}
\begin{equation*}
\log [\U] =\sum _{i=1}^{3}\log[\lambda_{i}](\overrightarrow{u}_{i}\otimes \overrightarrow{u}_{i})
\end{equation*}
%@END

%@DOC:SymTensor2::polarLapack(SymTensor2 U, Tensor2 R)
%Warning :
%This area is an automatic documentation generated from the DynELA source code.
%Do not change anything in this latex file between this position and the @END keyword.
\textcolor{purple}{\textbf{SymTensor2::polarLapack(SymTensor2 U, Tensor2 R)}}\label{SymTensor2::polarLapack(SymTensor2 U, Tensor2 R)}\index[DL]{SymTensor2!polarLapack(SymTensor2 U, Tensor2 R)}\\
Polar decomposition of a second order tensor using the Jacobi algorithm.\\ \hspace*{10mm}$\hookrightarrow$ SymTensor2 and Tensor2

\begin{tcolorbox}[width=\textwidth,myArgs,tabularx={ll|R},title=Arguments of SymTensor2::polarLapack]
SymTensor2&U&Symmetric tensor $\log[\U]$\\
Tensor2&R&Rotation tensor $\R$
\end{tcolorbox}

This method computes the polar decomposition of a second order tensor $\F$ and returns the symmetric tensor $\R$ and the tensor $\U$ so that:
\begin{equation*}
\F = \R \cdot \U
\end{equation*}
It uses the LAPACKE\_dgeev function of the Lapack library which is far from efficient for a trivial 3x3 matrix. So this method is very slow.
%@END

%@DOC:SymTensor2::polarLapackLnU(SymTensor2 U, Tensor2 R)
%Warning :
%This area is an automatic documentation generated from the DynELA source code.
%Do not change anything in this latex file between this position and the @END keyword.
\textcolor{purple}{\textbf{SymTensor2::polarLapackLnU(SymTensor2 U, Tensor2 R)}}\label{SymTensor2::polarLapackLnU(SymTensor2 U, Tensor2 R)}\index[DL]{SymTensor2!polarLapackLnU(SymTensor2 U, Tensor2 R)}\\
Polar decomposition of a second order tensor using the Jacobi algorithm.\\ \hspace*{10mm}$\hookrightarrow$ SymTensor2 and Tensor2

\begin{tcolorbox}[width=\textwidth,myArgs,tabularx={ll|R},title=Arguments of SymTensor2::polarLapackLnU]
SymTensor2&U&Symmetric tensor $\log[\U]$\\
Tensor2&R&Rotation tensor $\R$
\end{tcolorbox}

This method computes the polar decomposition of a second order tensor $\F$ and returns the symmetric tensor $\R$ and the tensor $\log[\U]$ so that:
\begin{equation*}
\F = \R \cdot \U
\end{equation*}
\begin{equation*}
\log [\U] =\sum _{i=1}^{3}\log[\lambda_{i}](\overrightarrow{u}_{i}\otimes \overrightarrow{u}_{i})
\end{equation*}
It uses the LAPACKE\_dgeev function of the Lapack library which is far from efficient for a trivial 3x3 matrix. So this method is very slow.
%@END

%@DOC:SymTensor2::polar(SymTensor2 U, Tensor2 R)
%Warning :
%This area is an automatic documentation generated from the DynELA source code.
%Do not change anything in this latex file between this position and the @END keyword.
\textcolor{purple}{\textbf{SymTensor2::polar(SymTensor2 U, Tensor2 R)}}\label{SymTensor2::polar(SymTensor2 U, Tensor2 R)}\index[DL]{SymTensor2!polar(SymTensor2 U, Tensor2 R)}\\
Polar decomposition of a second order tensor using the old \DynELA algorithm.\\ \hspace*{10mm}$\hookrightarrow$ SymTensor2 and Tensor2

\begin{tcolorbox}[width=\textwidth,myArgs,tabularx={ll|R},title=Arguments of SymTensor2::polar]
SymTensor2&U&Symmetric tensor $\log[\U]$\\
Tensor2&R&Rotation tensor $\R$
\end{tcolorbox}

This method computes the polar decomposition of a second order tensor $\F$ and returns the symmetric tensor $\R$ and the tensor $\U$ so that:
\begin{equation*}
\F = \R \cdot \U
\end{equation*}
%@END

%@DOC:SymTensor2::polarLnU(SymTensor2 U, Tensor2 R)
%Warning :
%This area is an automatic documentation generated from the DynELA source code.
%Do not change anything in this latex file between this position and the @END keyword.
\textcolor{purple}{\textbf{SymTensor2::polarLnU(SymTensor2 U, Tensor2 R)}}\label{SymTensor2::polarLnU(SymTensor2 U, Tensor2 R)}\index[DL]{SymTensor2!polarLnU(SymTensor2 U, Tensor2 R)}\\
Polar decomposition of a second order tensor using the old \DynELA algorithm.\\ \hspace*{10mm}$\hookrightarrow$ SymTensor2 and Tensor2

\begin{tcolorbox}[width=\textwidth,myArgs,tabularx={ll|R},title=Arguments of SymTensor2::polarLnU]
SymTensor2&U&Symmetric tensor $\log[\U]$\\
Tensor2&R&Rotation tensor $\R$
\end{tcolorbox}

This method computes the polar decomposition of a second order tensor $\F$ and returns the symmetric tensor $\R$ and the tensor $\log[\U]$ so that:
\begin{equation*}
\F = \R \cdot \U
\end{equation*}
\begin{equation*}
\log [\U] =\sum _{i=1}^{3}\log[\lambda_{i}](\overrightarrow{u}_{i}\otimes \overrightarrow{u}_{i})
\end{equation*}
%@END

\begin{tcolorbox}[width=0.95\textwidth,myTab,tabularx={l||C|C|C},title=Performance of the polar algorithms for symmetric tensors]%,boxrule=0.5pt]
 & $\U$ version& $\log[\U]$ & Precision\\
 & CPU (ns) & CPU (ns) & $\F-\R\cdot\U$\\\hline\hline
polarLapack & $2441$ & $2151$& $1.4210\times10^{-14}$ \\\hline
polar & $435$ & $436$& $3.5527\times10^{-15}$\\\hline
polarQL & $260$ & $281$& $1.0658\times10^{-14}$\\\hline
polarJacobi & $350$ & $356$ & $5.3291\times10^{-15}$\\\hline
polarCuppen & $206$ & $233$& $3.5527\times10^{-15}$
\end{tcolorbox}

\section{The Tensor3 library}

The Tensor3 library is used to store third order tensors defined in the \DynELA.

%@DOC:Tensor3::Tensor3()
%Warning :
%This area is an automatic documentation generated from the DynELA source code.
%Do not change anything in this latex file between this position and the @END keyword.
\textcolor{purple}{\textbf{Tensor3::Tensor3(~)}}\label{Tensor3::Tensor3()}\index[DL]{Tensor3!Tensor3(~)}\\
This method is the default constructor of a third order tensor.\\ \hspace*{10mm}$\hookrightarrow$ Tensor3

All components are initialized to zero by default.
%@END

%@DOC:Tensor3::~Tensor3()
%Warning :
%This area is an automatic documentation generated from the DynELA source code.
%Do not change anything in this latex file between this position and the @END keyword.
\textcolor{purple}{\textbf{Tensor3::$\sim$Tensor3(~)}}\label{Tensor3::~Tensor3()}\index[DL]{Tensor3!$\sim$Tensor3(~)}\\
Destructor of the Tensor3 class.

%@END

%@DOC:Tensor3::setToUnity()
%Warning :
%This area is an automatic documentation generated from the DynELA source code.
%Do not change anything in this latex file between this position and the @END keyword.
\textcolor{purple}{\textbf{Tensor3::setToUnity(~)}}\label{Tensor3::setToUnity()}\index[DL]{Tensor3!setToUnity(~)}\\
Returns an identity third order tensor.

\hspace*{10mm}\textcolor{red}{\textbf{Warning :} This method modifies its own argument}

This method transforms the current tensor to a unity tensor.
%@END

\section{The Tensor4 library}

The Tensor4 library is used to store fourth order tensors defined in the \DynELA.

%@DOC:Tensor4::Tensor4()
%Warning :
%This area is an automatic documentation generated from the DynELA source code.
%Do not change anything in this latex file between this position and the @END keyword.
\textcolor{purple}{\textbf{Tensor4::Tensor4(~)}}\label{Tensor4::Tensor4()}\index[DL]{Tensor4!Tensor4(~)}\\
Default constructor of the TenTensor4sor2 class.\\ \hspace*{10mm}$\hookrightarrow$ Tensor4

All components are initialized to zero by default.
%@END

%@DOC:Tensor4::~Tensor4()
%Warning :
%This area is an automatic documentation generated from the DynELA source code.
%Do not change anything in this latex file between this position and the @END keyword.
\textcolor{purple}{\textbf{Tensor4::$\sim$Tensor4(~)}}\label{Tensor4::~Tensor4()}\index[DL]{Tensor4!$\sim$Tensor4(~)}\\
Destructor of the Tensor4 class.

%@END

%@DOC:Tensor4::setToUnity()
%Warning :
%This area is an automatic documentation generated from the DynELA source code.
%Do not change anything in this latex file between this position and the @END keyword.
\textcolor{purple}{\textbf{Tensor4::setToUnity(~)}}\label{Tensor4::setToUnity()}\index[DL]{Tensor4!setToUnity(~)}\\
Unity tensor.

\hspace*{10mm}\textcolor{red}{\textbf{Warning :} This method modifies its own argument}

This method transforms the current tensor to a unity tensor.
%@END
