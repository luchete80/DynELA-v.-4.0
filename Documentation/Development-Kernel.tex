% !TeX spellcheck = en_US
% !TeX root = DynELA.tex
%
% LaTeX source file of DynELA FEM Code
%
% (c) by Olivier Pantalé 2020
%
\chapter{DynELA Kernel library}

\startcontents[chapters]
\printmyminitoc[2]\LETTRINE{T}he \DynELA~Kernel library is the lowest library of all the \DynELA~libraries. It contains basic classes that allow the definition of the non-mathematical functions of the \DynELA. We find in this library the List class allowing to manage the objects of the \DynELA, the String classes, the basic options, the management of the log files,...

\section{The MacAdresses class}



%@DOC:MacAddresses::MacAddresses()
% Automatic documentation generated from the DynELA source code
% Do not change anything in this LaTeX file between the @DOC and the @END keywords.
\textcolor{purple}{\textbf{MacAddresses::MacAddresses(~)}}\label{MacAddresses::MacAddresses()}\index[DL]{MacAddresses!MacAddresses(~)}\\
Default constructor of the MacAddresses class.\vspace*{-0.5em}
\begin{tcolorbox}[grow to left by=-1cm, width=\textwidth-1cm,myArgs,tabularx={l|R}]
$\hookrightarrow$ MacAddresses&
\end{tcolorbox}

%@END

\subsection{Basic methods}

%@DOC:MacAddresses::getAddress(int n)
% Automatic documentation generated from the DynELA source code
% Do not change anything in this LaTeX file between the @DOC and the @END keywords.
\textcolor{purple}{\textbf{MacAddresses::getAddress(int n)}}\label{MacAddresses::getAddress(int n)}\index[DL]{MacAddresses!getAddress(int n)}\\
Return the Mac Address \#n from the list.\vspace*{-0.5em}
\begin{tcolorbox}[grow to left by=-1cm, width=\textwidth-1cm,myArgs,tabularx={l|R}]
$\hookrightarrow$ String&
\end{tcolorbox}

\begin{tcolorbox}[width=\textwidth,myArgs,tabularx={ll|R}]
int&n&The index of the MacAddresses to return.
\end{tcolorbox}

%@END

%@DOC:MacAddresses::getInterface(int n)
% Automatic documentation generated from the DynELA source code
% Do not change anything in this LaTeX file between the @DOC and the @END keywords.
\textcolor{purple}{\textbf{MacAddresses::getInterface(int n)}}\label{MacAddresses::getInterface(int n)}\index[DL]{MacAddresses!getInterface(int n)}\\
Return the Internet interface \#n from the list.\vspace*{-0.5em}
\begin{tcolorbox}[grow to left by=-1cm, width=\textwidth-1cm,myArgs,tabularx={l|R}]
$\hookrightarrow$ String&
\end{tcolorbox}

\begin{tcolorbox}[width=\textwidth,myArgs,tabularx={ll|R}]
int&n&The index of the Internet interface to return.
\end{tcolorbox}

%@END

%@DOC:MacAddresses::getNumber()
% Automatic documentation generated from the DynELA source code
% Do not change anything in this LaTeX file between the @DOC and the @END keywords.
\textcolor{purple}{\textbf{MacAddresses::getNumber(~)}}\label{MacAddresses::getNumber()}\index[DL]{MacAddresses!getNumber(~)}\\
Return the number of Internet interfaces of the computer.\vspace*{-0.5em}
\begin{tcolorbox}[grow to left by=-1cm, width=\textwidth-1cm,myArgs,tabularx={l|R}]
$\hookrightarrow$ int&
\end{tcolorbox}

%@END

\section{The Settings class}

%@DOC:Settings::Settings
% Automatic documentation generated from the DynELA source code
% Do not change anything in this LaTeX file between the @DOC and the @END keywords.
\textcolor{purple}{\textbf{Settings::Settings}}\label{Settings::Settings}\index[DL]{Settings!Settings}\\
Settings class.

This class is used to store settings information for \DynELA.
%@END



%@DOC:Settings::Settings()
% Automatic documentation generated from the DynELA source code
% Do not change anything in this LaTeX file between the @DOC and the @END keywords.
\textcolor{purple}{\textbf{Settings::Settings(~)}}\label{Settings::Settings()}\index[DL]{Settings!Settings(~)}\\
Constructor of the Settings class.\vspace*{-0.5em}
\begin{tcolorbox}[grow to left by=-1cm, width=\textwidth-1cm,myArgs,tabularx={l|R}]
$\hookrightarrow$ Settings&
\end{tcolorbox}

%@END

\subsection{Basic methods}

%@DOC:Settings::load(string name)
% Automatic documentation generated from the DynELA source code
% Do not change anything in this LaTeX file between the @DOC and the @END keywords.
\textcolor{purple}{\textbf{Settings::load(string name)}}\label{Settings::load(string name)}\index[DL]{Settings!load(string name)}\\
Load settings from a file.\vspace*{-0.5em}
\begin{tcolorbox}[grow to left by=-1cm, width=\textwidth-1cm,myArgs,tabularx={l|R}]
$\hookrightarrow$ bool&
\end{tcolorbox}

\begin{tcolorbox}[width=\textwidth,myArgs,tabularx={ll|R}]
string & name & Name of the file to read settings from
\end{tcolorbox}

%@END

%@DOC:Settings::save()
% Automatic documentation generated from the DynELA source code
% Do not change anything in this LaTeX file between the @DOC and the @END keywords.
\textcolor{purple}{\textbf{Settings::save(~)}}\label{Settings::save()}\index[DL]{Settings!save(~)}\\
Saves settings.\vspace*{-0.5em}
\begin{tcolorbox}[grow to left by=-1cm, width=\textwidth-1cm,myArgs,tabularx={l|R}]
$\hookrightarrow$ bool&
\end{tcolorbox}

%@END

%@DOC:Settings::dump()
% Automatic documentation generated from the DynELA source code
% Do not change anything in this LaTeX file between the @DOC and the @END keywords.
\textcolor{purple}{\textbf{Settings::dump(~)}}\label{Settings::dump()}\index[DL]{Settings!dump(~)}\\
Dumps the conttent of the settings file to the output.

%@END

%@DOC:Settings::isChanged()
% Automatic documentation generated from the DynELA source code
% Do not change anything in this LaTeX file between the @DOC and the @END keywords.
\textcolor{purple}{\textbf{Settings::isChanged(~)}}\label{Settings::isChanged()}\index[DL]{Settings!isChanged(~)}\\
Checks if settings has been changed and needs rewrite.\vspace*{-0.5em}
\begin{tcolorbox}[grow to left by=-1cm, width=\textwidth-1cm,myArgs,tabularx={l|R}]
$\hookrightarrow$ bool&
\end{tcolorbox}

%@END

\section{The System class}

%@DOC:System::System
% Automatic documentation generated from the DynELA source code
% Do not change anything in this LaTeX file between the @DOC and the @END keywords.
\textcolor{purple}{\textbf{System::System}}\label{System::System}\index[DL]{System!System}\\
System class.

This class is used to store system information for \DynELA.
%@END

%@DOC:System::execute(String s)
% Automatic documentation generated from the DynELA source code
% Do not change anything in this LaTeX file between the @DOC and the @END keywords.
\textcolor{purple}{\textbf{System::execute(String s)}}\label{System::execute(String s)}\index[DL]{System!execute(String s)}\\
Execution of a system command\vspace*{-0.5em}
\begin{tcolorbox}[grow to left by=-1cm, width=\textwidth-1cm,myArgs,tabularx={l|R}]
$\hookrightarrow$ int & Status of the system command. A value of 0 is returned if everything works well, another value is returned if a problem was encountered during execution.
\end{tcolorbox}

\begin{tcolorbox}[width=\textwidth,myArgs,tabularx={ll|R}]
String & s & System command to launch.
\end{tcolorbox}

This method executes an external command. Launching a new process through the system() command.
The new program is a new process totally independent from the current application.
The only link is that the calling program waits for the end of the execution of the new process to continue the operations (except if we use the magic parameter \& in the command line...).
Do you know System?). As this command is simple, the working environment for the execution of the new process is /bin/sh. An error message is generated if a problem is encountered during execution.
For more complex situations, it will be necessary to use the "classic" fork() and exec() program launch methods. But this inevitably leads to threads problems and that's another story actually ;-0
%@END

%@DOC:System::env(String s)
% Automatic documentation generated from the DynELA source code
% Do not change anything in this LaTeX file between the @DOC and the @END keywords.
\textcolor{purple}{\textbf{System::env(String s)}}\label{System::env(String s)}\index[DL]{System!env(String s)}\\
Retrieve the value associated with an environment variable\vspace*{-0.5em}
\begin{tcolorbox}[grow to left by=-1cm, width=\textwidth-1cm,myArgs,tabularx={l|R}]
$\hookrightarrow$ String & Value associated with the environment variable in the form of String
\end{tcolorbox}

\begin{tcolorbox}[width=\textwidth,myArgs,tabularx={ll|R}]
String & s & Environment variable.
\end{tcolorbox}

This method retrieves the value associated with a System environment variable. If this variable is not defined, this method returns the following string "cannot get environment value". The returned value is of type String.
%@END

%@DOC:System::existEnv(String s)
% Automatic documentation generated from the DynELA source code
% Do not change anything in this LaTeX file between the @DOC and the @END keywords.
\textcolor{purple}{\textbf{System::existEnv(String s)}}\label{System::existEnv(String s)}\index[DL]{System!existEnv(String s)}\\
Tests for the presence of a defined environment variable\vspace*{-0.5em}
\begin{tcolorbox}[grow to left by=-1cm, width=\textwidth-1cm,myArgs,tabularx={l|R}]
$\hookrightarrow$ bool & true if the environment variable is set on the system, false if not.
\end{tcolorbox}

\begin{tcolorbox}[width=\textwidth,myArgs,tabularx={ll|R}]
String & s & Environment variable.
\end{tcolorbox}

This method tests the definition of an environment variable. It returns a boolean value that indicates the state of definition of this environment variable.
%@END

%@DOC:System::login()
% Automatic documentation generated from the DynELA source code
% Do not change anything in this LaTeX file between the @DOC and the @END keywords.
\textcolor{purple}{\textbf{System::login(~)}}\label{System::login()}\index[DL]{System!login(~)}\\
Returns the user's login.\vspace*{-0.5em}
\begin{tcolorbox}[grow to left by=-1cm, width=\textwidth-1cm,myArgs,tabularx={l|R}]
$\hookrightarrow$ String & The user's login or "unknown user" if this information cannot be given.
\end{tcolorbox}

This method returns the login of the system user in the form of a string.
%@END

%@DOC:System::hostname()
% Automatic documentation generated from the DynELA source code
% Do not change anything in this LaTeX file between the @DOC and the @END keywords.
\textcolor{purple}{\textbf{System::hostname(~)}}\label{System::hostname()}\index[DL]{System!hostname(~)}\\
Returns the machine name\vspace*{-0.5em}
\begin{tcolorbox}[grow to left by=-1cm, width=\textwidth-1cm,myArgs,tabularx={l|R}]
$\hookrightarrow$ String & Machine name or "unknown host" if this information cannot be given.
\end{tcolorbox}

This method returns the name of the machine on which the application is running.
%@END

%@DOC:System::getDate(bool b)
% Automatic documentation generated from the DynELA source code
% Do not change anything in this LaTeX file between the @DOC and the @END keywords.
\textcolor{purple}{\textbf{System::getDate(bool b)}}\label{System::getDate(bool b)}\index[DL]{System!getDate(bool b)}\\
Returns the current date and time\vspace*{-0.5em}
\begin{tcolorbox}[grow to left by=-1cm, width=\textwidth-1cm,myArgs,tabularx={l|R}]
$\hookrightarrow$ String & Current date and time or "unknown date" if this information cannot be given.
\end{tcolorbox}

\begin{tcolorbox}[width=\textwidth,myArgs,tabularx={ll|R}]

\end{tcolorbox}

This method returns the current date and time at the system level.
  - full this boolean value defines the nature of the result returned. If the value is true then the format is the full format of the form (Fri Jan 25 15:08:24 2002) if the value is false then the format returned is the short format of the form (Jan 25, 2002). The default value if nothing is accurate is true.
%@END

%@DOC:System::getTime()
% Automatic documentation generated from the DynELA source code
% Do not change anything in this LaTeX file between the @DOC and the @END keywords.
\textcolor{purple}{\textbf{System::getTime(~)}}\label{System::getTime()}\index[DL]{System!getTime(~)}\\
Returns the current time\vspace*{-0.5em}
\begin{tcolorbox}[grow to left by=-1cm, width=\textwidth-1cm,myArgs,tabularx={l|R}]
$\hookrightarrow$ String & Vlue of the current time as a String (format: 14:23:26)
\end{tcolorbox}

This method returns the current time to the system level.
%@END

%@DOC:System::pathname()
% Automatic documentation generated from the DynELA source code
% Do not change anything in this LaTeX file between the @DOC and the @END keywords.
\textcolor{purple}{\textbf{System::pathname(~)}}\label{System::pathname()}\index[DL]{System!pathname(~)}\\
Returns the name of the current directory\vspace*{-0.5em}
\begin{tcolorbox}[grow to left by=-1cm, width=\textwidth-1cm,myArgs,tabularx={l|R}]
$\hookrightarrow$ String & Current directory or "unknown pathname" if this information cannot be given.
\end{tcolorbox}

This method returns the name of the current directory from which the execution was started.
%@END

%@DOC:System::execPath()
% Automatic documentation generated from the DynELA source code
% Do not change anything in this LaTeX file between the @DOC and the @END keywords.
\textcolor{purple}{\textbf{System::execPath(~)}}\label{System::execPath()}\index[DL]{System!execPath(~)}\\
Returns the execution path of the current application path\vspace*{-0.5em}
\begin{tcolorbox}[grow to left by=-1cm, width=\textwidth-1cm,myArgs,tabularx={l|R}]
$\hookrightarrow$ String & Current path or "unknown pathname" if this information cannot be given.
\end{tcolorbox}

This method returns the name of the current application path from which the execution was started.
%@END

%@DOC:System::hostID()
% Automatic documentation generated from the DynELA source code
% Do not change anything in this LaTeX file between the @DOC and the @END keywords.
\textcolor{purple}{\textbf{System::hostID(~)}}\label{System::hostID()}\index[DL]{System!hostID(~)}\\
Returns the host back from the machine\vspace*{-0.5em}
\begin{tcolorbox}[grow to left by=-1cm, width=\textwidth-1cm,myArgs,tabularx={l|R}]
$\hookrightarrow$ String & hostId of the machine.
\end{tcolorbox}

This method returns the host of the machine on which the program is executed. The host is an integer value usually given in hexadecimal form 0xFFFFFFFFFFFFFF on 32 bits. This number is unique per machine.
%@END

\section{The List class}

The List class is a basic class in the \DynELA. It is even at the origin of all the developments made in finite element and numerical computation since 1992. Its development is at the heart of the PostCopo and F3R applications which were realized during my thesis work between 1992 and 1996. Originally, it was written in C language and was thus a dynamic list of pointers on \textbf{void$\star$} which was violently (and yes I assume this) casted, during the definition of the list, into pointers on the real elements of the list. The automatic dynamic memory allocation mechanism for the list in increments of $10$, in general, was already present in the first version of this list in C. Later, with the advent of \Cpp~, this list was switched to its current form with the template mechanism provided by the \Cpp~language.

%@DOC:List::List
% Automatic documentation generated from the DynELA source code
% Do not change anything in this LaTeX file between the @DOC and the @END keywords.
\textcolor{purple}{\textbf{List::List}}\label{List::List}\index[DL]{List!List}\\
Management of objects as List.

This class is used to store all type of object and manipulate them as a list (for example: list of Nodes, Elements, Boundary conditions,...).
This List is a dynamic one, the initialization is performed with a default stack size defined by DEFAULT\_stack\_size, as soon as there is no more space left to store a new object,
the List size is increased with respect to the DEFAULT\_stack\_inc value.
%@END



%@DOC:List::List(long size)
% Automatic documentation generated from the DynELA source code
% Do not change anything in this LaTeX file between the @DOC and the @END keywords.
\textcolor{purple}{\textbf{List::List(long size)}}\label{List::List(long size)}\index[DL]{List!List(long size)}\\
Constructor of the List class.

\begin{tcolorbox}[width=\textwidth,myArgs,tabularx={ll|R}]
long & size & Initial size of the list (default value DEFAULT\_stack\_inc).
\end{tcolorbox}

This constructor allocates the default memory for an instance of the List class.
If the size of the list is not specified, the default size is taken into account, which is defined by the value of DEFAULT\_stack\_size.
%@END

\subsection{Basic methods}

%@DOC:List::redim(long size)
% Automatic documentation generated from the DynELA source code
% Do not change anything in this LaTeX file between the @DOC and the @END keywords.
\textcolor{purple}{\textbf{List::redim(long size)}}\label{List::redim(long size)}\index[DL]{List!redim(long size)}\\
Resize the storage space of a List.

\begin{tcolorbox}[width=\textwidth,myArgs,tabularx={ll|R}]
long & size & New size of the list.
\end{tcolorbox}

This method is used to increase or decrease the size of a list.
If the proposed new size is smaller than the minimum size needed to store the current elements of the list, an error is generated.\\
This method should generally not be called by the user (unless the user has mastered the trick).
This method is heavily used internally by the other methods of the class.
In case the user does not master this kind of operation sufficiently, it is better to let the class manage its own memory allocations.\\
A possible use of this method is memory pre-allocation, when the number of objects that will be stored in the list is known in advance.
The size of the list is then adjusted to this value, which avoids dynamic size adjustment operations that take CPU time.
Of course, dynamic allocation mechanisms exist and you can exceed this value.
%@END

%@DOC:List::close()
% Automatic documentation generated from the DynELA source code
% Do not change anything in this LaTeX file between the @DOC and the @END keywords.
\textcolor{purple}{\textbf{List::close(~)}}\label{List::close()}\index[DL]{List!close(~)}\\
This method closes the List, i.e. resize the stack size to the number of elements to save memory.

This method is used to adjust the size of the list according to the number of real objects in the list.
This method allows to recover memory space, mainly for small lists.
%@END

%@DOC:List::operator()(long index)
% Automatic documentation generated from the DynELA source code
% Do not change anything in this LaTeX file between the @DOC and the @END keywords.
\textcolor{purple}{\textbf{List::operator(~)(long index)}}\label{List::operator()(long index)}\index[DL]{List!operator(~)(long index)}\\
This is the accessor to the elements of the list.

\begin{tcolorbox}[width=\textwidth,myArgs,tabularx={ll|R}]
long & index & Index of the element to get in the list.
\end{tcolorbox}

This method is used to access items on the list.
This access is both read and write.
This method returns element [i] of the list.
The baseline is 0 (first element of index 0) as usual in C and C++.
%@END

%@DOC:List::initLoop()
% Automatic documentation generated from the DynELA source code
% Do not change anything in this LaTeX file between the @DOC and the @END keywords.
\textcolor{purple}{\textbf{List::initLoop(~)}}\label{List::initLoop()}\index[DL]{List!initLoop(~)}\\
Finalization of a loop on the list.

This method has to be called after a loop using one of the iterator methods for the List.
%@END

%@DOC:List::next()
% Automatic documentation generated from the DynELA source code
% Do not change anything in this LaTeX file between the @DOC and the @END keywords.
\textcolor{purple}{\textbf{List::next(~)}}\label{List::next()}\index[DL]{List!next(~)}\\
Next element in the list.\vspace*{-0.5em}
\begin{tcolorbox}[grow to left by=-1cm, width=\textwidth-1cm,myArgs,tabularx={l|R}]
$\hookrightarrow$ Type & Next item in the list or NULL if it does not exist.
\end{tcolorbox}

This method uses an internal list lookup mechanism to return the next element in the list.
To use this method, it is necessary to define the list boundaries, and to have the start referenced by the first(), last() or accessors() methods.
%@END

%@DOC:List::currentUp()
% Automatic documentation generated from the DynELA source code
% Do not change anything in this LaTeX file between the @DOC and the @END keywords.
\textcolor{purple}{\textbf{List::currentUp(~)}}\label{List::currentUp()}\index[DL]{List!currentUp(~)}\\
Next element in the list.\vspace*{-0.5em}
\begin{tcolorbox}[grow to left by=-1cm, width=\textwidth-1cm,myArgs,tabularx={l|R}]
$\hookrightarrow$ Type & Next item in the list or NULL if it does not exist.
\end{tcolorbox}

This method uses an internal list lookup mechanism to return the element following the previous call in the list.
To use this method, it is necessary to define the list boundaries, and to have the start referenced by the first(), last() or accessors() methods.
%@END

%@DOC:List::currentDown()
% Automatic documentation generated from the DynELA source code
% Do not change anything in this LaTeX file between the @DOC and the @END keywords.
\textcolor{purple}{\textbf{List::currentDown(~)}}\label{List::currentDown()}\index[DL]{List!currentDown(~)}\\
Previous element in the list.\vspace*{-0.5em}
\begin{tcolorbox}[grow to left by=-1cm, width=\textwidth-1cm,myArgs,tabularx={l|R}]
$\hookrightarrow$ Type & Previous item in the list or NULL if it does not exist.
\end{tcolorbox}

This method uses an internal list lookup mechanism to return the element preceding the previous call in the list.
To use this method, it is necessary to define the list boundaries, and to have the start referenced by the first(), last() or accessors() methods.
%@END

%@DOC:List::first()
% Automatic documentation generated from the DynELA source code
% Do not change anything in this LaTeX file between the @DOC and the @END keywords.
\textcolor{purple}{\textbf{List::first(~)}}\label{List::first()}\index[DL]{List!first(~)}\\
First element in the list.\vspace*{-0.5em}
\begin{tcolorbox}[grow to left by=-1cm, width=\textwidth-1cm,myArgs,tabularx={l|R}]
$\hookrightarrow$ Type & First item in the list or NULL if it does not exist.
\end{tcolorbox}

%@END

%@DOC:List::last()
% Automatic documentation generated from the DynELA source code
% Do not change anything in this LaTeX file between the @DOC and the @END keywords.
\textcolor{purple}{\textbf{List::last(~)}}\label{List::last()}\index[DL]{List!last(~)}\\
Last element in the list.\vspace*{-0.5em}
\begin{tcolorbox}[grow to left by=-1cm, width=\textwidth-1cm,myArgs,tabularx={l|R}]
$\hookrightarrow$ Type & Last item in the list or NULL if it does not exist.
\end{tcolorbox}

%@END

%@DOC:List::previous()
% Automatic documentation generated from the DynELA source code
% Do not change anything in this LaTeX file between the @DOC and the @END keywords.
\textcolor{purple}{\textbf{List::previous(~)}}\label{List::previous()}\index[DL]{List!previous(~)}\\
Previous element in the list.\vspace*{-0.5em}
\begin{tcolorbox}[grow to left by=-1cm, width=\textwidth-1cm,myArgs,tabularx={l|R}]
$\hookrightarrow$ Type & Previous item in the list or NULL if it does not exist.
\end{tcolorbox}

This method uses an internal list lookup mechanism to return the element preceding the previous call in the list.
To use this method, it is necessary to define the list boundaries, and to have the start referenced by the first(), last() or accessors() methods.
%@END

%@DOC:List::current()
% Automatic documentation generated from the DynELA source code
% Do not change anything in this LaTeX file between the @DOC and the @END keywords.
\textcolor{purple}{\textbf{List::current(~)}}\label{List::current()}\index[DL]{List!current(~)}\\
Current element in the list.\vspace*{-0.5em}
\begin{tcolorbox}[grow to left by=-1cm, width=\textwidth-1cm,myArgs,tabularx={l|R}]
$\hookrightarrow$ Type & Current item in the list or NULL if it does not exist.
\end{tcolorbox}

This method uses an internal list lookup mechanism to return the current element in the list.
To use this method, it is necessary to define the list boundaries, and to have the start referenced by the first(), last() or accessors() methods.
%@END

%@DOC:List::size()
% Automatic documentation generated from the DynELA source code
% Do not change anything in this LaTeX file between the @DOC and the @END keywords.
\textcolor{purple}{\textbf{List::size(~)}}\label{List::size()}\index[DL]{List!size(~)}\\
Current size of the list.\vspace*{-0.5em}
\begin{tcolorbox}[grow to left by=-1cm, width=\textwidth-1cm,myArgs,tabularx={l|R}]
$\hookrightarrow$ long & Size
\end{tcolorbox}

%@END

%@DOC:List::stack()
% Automatic documentation generated from the DynELA source code
% Do not change anything in this LaTeX file between the @DOC and the @END keywords.
\textcolor{purple}{\textbf{List::stack(~)}}\label{List::stack()}\index[DL]{List!stack(~)}\\
Stack size of the list.\vspace*{-0.5em}
\begin{tcolorbox}[grow to left by=-1cm, width=\textwidth-1cm,myArgs,tabularx={l|R}]
$\hookrightarrow$ long&
\end{tcolorbox}

%@END

%@DOC:List::stackIncrement()
% Automatic documentation generated from the DynELA source code
% Do not change anything in this LaTeX file between the @DOC and the @END keywords.
\textcolor{purple}{\textbf{List::stackIncrement(~)}}\label{List::stackIncrement()}\index[DL]{List!stackIncrement(~)}\\
Stack increment size of the list.\vspace*{-0.5em}
\begin{tcolorbox}[grow to left by=-1cm, width=\textwidth-1cm,myArgs,tabularx={l|R}]
$\hookrightarrow$ long&
\end{tcolorbox}

%@END

%@DOC:List::flush()
% Automatic documentation generated from the DynELA source code
% Do not change anything in this LaTeX file between the @DOC and the @END keywords.
\textcolor{purple}{\textbf{List::flush(~)}}\label{List::flush()}\index[DL]{List!flush(~)}\\
Empties the list.

This method empties the contents of the stack and returns its real size to zero and its stack size to DEFAULT\_stack\_size.
The stack is as good as new !!! (it's a rechargeable battery !)
%@END

%@DOC:List::operator<<(Type object)
% Automatic documentation generated from the DynELA source code
% Do not change anything in this LaTeX file between the @DOC and the @END keywords.
\textcolor{purple}{\textbf{List::operator<<(Type object)}}\label{List::operator<<(Type object)}\index[DL]{List!operator<<(Type object)}\\
Add an object to the list.

\begin{tcolorbox}[width=\textwidth,myArgs,tabularx={ll|R}]
Type & object & Object to add to the list.
\end{tcolorbox}

This method adds an object to the list. The object is added to the end of the list, and the list size is automatically incremented if necessary.
%@END

%@DOC:List::insert(Type object, long index)
% Automatic documentation generated from the DynELA source code
% Do not change anything in this LaTeX file between the @DOC and the @END keywords.
\textcolor{purple}{\textbf{List::insert(Type object, long index)}}\label{List::insert(Type object, long index)}\index[DL]{List!insert(Type object, long index)}\\
Insert an object in the list.

\begin{tcolorbox}[width=\textwidth,myArgs,tabularx={ll|R}]
Type & object & Object to insert into the list.\\
long & index & Defines the position of the insertion in the list..
\end{tcolorbox}

This method inserts an object to the list. The object is inserted at a given index in the list, and the list size is automatically incremented if necessary.
%@END

%@DOC:List::add(Type object)
% Automatic documentation generated from the DynELA source code
% Do not change anything in this LaTeX file between the @DOC and the @END keywords.
\textcolor{purple}{\textbf{List::add(Type object)}}\label{List::add(Type object)}\index[DL]{List!add(Type object)}\\
Add an object to the list.

\begin{tcolorbox}[width=\textwidth,myArgs,tabularx={ll|R}]
Type & object & Object to add to the list.
\end{tcolorbox}

This method adds an object to the list. The object is added to the end of the list, and the list size is automatically incremented if necessary.
%@END

%@DOC:List::inverse()
% Automatic documentation generated from the DynELA source code
% Do not change anything in this LaTeX file between the @DOC and the @END keywords.
\textcolor{purple}{\textbf{List::inverse(~)}}\label{List::inverse()}\index[DL]{List!inverse(~)}\\
Reverse the order of the list.

This method reverses the order of the elements in the list.
%@END

%@DOC:List::del(long start, long stop)
% Automatic documentation generated from the DynELA source code
% Do not change anything in this LaTeX file between the @DOC and the @END keywords.
\textcolor{purple}{\textbf{List::del(long start, long stop)}}\label{List::del(long start, long stop)}\index[DL]{List!del(long start, long stop)}\\
Removes a set of elements from the List.

\begin{tcolorbox}[width=\textwidth,myArgs,tabularx={ll|R}]
long & start & First element to suppress from the list.\\
long & stop & Last element to suppress from the list.
\end{tcolorbox}

This method removes a set of items from the list.
This method is used to remove an entire segment from the list, by defining the start and end indexes of the segment in the list.
If the start and stop parameters are equal, only one element is deleted.
%@END

%@DOC:List::del(long index)
% Automatic documentation generated from the DynELA source code
% Do not change anything in this LaTeX file between the @DOC and the @END keywords.
\textcolor{purple}{\textbf{List::del(long index)}}\label{List::del(long index)}\index[DL]{List!del(long index)}\\
Removes an element from the List.

\begin{tcolorbox}[width=\textwidth,myArgs,tabularx={ll|R}]
long & index & Index of the element to suppress from the list.
\end{tcolorbox}

This method removes an items from the list.
This method is used to remove an item from the list, by defining the index of the element in the list.
%@END

%@DOC:List::delBefore(long index)
% Automatic documentation generated from the DynELA source code
% Do not change anything in this LaTeX file between the @DOC and the @END keywords.
\textcolor{purple}{\textbf{List::delBefore(long index)}}\label{List::delBefore(long index)}\index[DL]{List!delBefore(long index)}\\
Removes a set of elements from the List.

\begin{tcolorbox}[width=\textwidth,myArgs,tabularx={ll|R}]
long & index & Index of the last element to suppress from the list.
\end{tcolorbox}

This method removes all items in the list between the beginning of the list and the value given as an argument to this method.
This method is equivalent to del(0, index-1).
%@END

%@DOC:List::delAfter(long index)
% Automatic documentation generated from the DynELA source code
% Do not change anything in this LaTeX file between the @DOC and the @END keywords.
\textcolor{purple}{\textbf{List::delAfter(long index)}}\label{List::delAfter(long index)}\index[DL]{List!delAfter(long index)}\\
Removes a set of elements from the List.

\begin{tcolorbox}[width=\textwidth,myArgs,tabularx={ll|R}]
long & index & Index of the first element to suppress from the list.
\end{tcolorbox}

This method deletes all items in the list between the value given as an argument to this method and the end of the list.
This method is equivalent to del(index+1, last()).
%@END

%@DOC:List::objectSize()
% Automatic documentation generated from the DynELA source code
% Do not change anything in this LaTeX file between the @DOC and the @END keywords.
\textcolor{purple}{\textbf{List::objectSize(~)}}\label{List::objectSize()}\index[DL]{List!objectSize(~)}\\
Memory size of a list.

This method return the memory size a list, i.e. the memory size of the stack and of the List itself.
%@END

%@DOC:List::print(ostream output)
% Automatic documentation generated from the DynELA source code
% Do not change anything in this LaTeX file between the @DOC and the @END keywords.
\textcolor{purple}{\textbf{List::print(ostream output)}}\label{List::print(ostream output)}\index[DL]{List!print(ostream output)}\\
Prints the content of a list.

\begin{tcolorbox}[width=\textwidth,myArgs,tabularx={ll|R}]
ostream & output & Output stream.
\end{tcolorbox}

%@END

%@DOC:List::print(Type object)
% Automatic documentation generated from the DynELA source code
% Do not change anything in this LaTeX file between the @DOC and the @END keywords.
\textcolor{purple}{\textbf{List::print(Type object)}}\label{List::print(Type object)}\index[DL]{List!print(Type object)}\\
Search an object in the List.\vspace*{-0.5em}
\begin{tcolorbox}[grow to left by=-1cm, width=\textwidth-1cm,myArgs,tabularx={l|R}]
$\hookrightarrow$ Type&
\end{tcolorbox}

\begin{tcolorbox}[width=\textwidth,myArgs,tabularx={ll|R}]
Type & object & Object to search.
\end{tcolorbox}

This method performs a simple search for an item in the list and returns an Index indicating the place of the object in the list.
If the object is not found, it returns the value -1.
%@END

%@DOC:List::contains(Type object)
% Automatic documentation generated from the DynELA source code
% Do not change anything in this LaTeX file between the @DOC and the @END keywords.
\textcolor{purple}{\textbf{List::contains(Type object)}}\label{List::contains(Type object)}\index[DL]{List!contains(Type object)}\\
Search an object in the List.\vspace*{-0.5em}
\begin{tcolorbox}[grow to left by=-1cm, width=\textwidth-1cm,myArgs,tabularx={l|R}]
$\hookrightarrow$ bool&
\end{tcolorbox}

\begin{tcolorbox}[width=\textwidth,myArgs,tabularx={ll|R}]
Type & object & Object to search.
\end{tcolorbox}

This method performs a simple search for an item in the list and returns a boolean according to the presence or or not of this object in the list.
%@END

%@DOC:List::sort(bool (*f)(Type, Type))
% Automatic documentation generated from the DynELA source code
% Do not change anything in this LaTeX file between the @DOC and the @END keywords.
\textcolor{purple}{\textbf{List::sort(bool ($\star$f)(Type, Type))}}\label{List::sort(bool (*f)(Type, Type))}\index[DL]{List!sort(bool ($\star$f)(Type, Type))}\\
Sort the list using a comparison function.

\begin{tcolorbox}[width=\textwidth,myArgs,tabularx={ll|R}]
bool & (*f) & Function defined in the objects to sort the elements.
\end{tcolorbox}

This method sorts the elements of the stack using a user-defined comparison function.
This method is very powerful for sorting a list and very flexible in use.
The usage may seem complex, but it is defined in the example below.
\begin{CppListing}
class truc
{
  public:
  double z; // a value
};
List <truc*> listeTrucs; // the list
bool compare(truc* p1,truc* p2) // the comparing function
{
  return (p1->z < p2->z); // comparison
}
...
{
...
listeTrucs.sort(compare); // sorts the list using the comparison function
}
\end{CppListing}
%@END

\section{The ListIndex class}

The ListIndex class is an extension of the List class, where objects contain an internal number used to manage the objects inside the list.

%@DOC:ListIndex::ListIndex
% Automatic documentation generated from the DynELA source code
% Do not change anything in this LaTeX file between the @DOC and the @END keywords.
\textcolor{purple}{\textbf{ListIndex::ListIndex}}\label{ListIndex::ListIndex}\index[DL]{ListIndex!ListIndex}\\
Management of objects as ListIndex.

This class is used to store all type of object and manipulate them as a listIndex (for example: listIndex of Nodes, Elements, Boundary conditions,...).
This ListIndex is a dynamic one, the initialization is performed with a default stack size defined by DEFAULT\_stack\_size, as soon as there is no more space left to store a new object,
the ListIndex size is increased with respect to the DEFAULT\_stack\_inc value.
%@END



%@DOC:ListIndex::ListIndex(long size)
% Automatic documentation generated from the DynELA source code
% Do not change anything in this LaTeX file between the @DOC and the @END keywords.
\textcolor{purple}{\textbf{ListIndex::ListIndex(long size)}}\label{ListIndex::ListIndex(long size)}\index[DL]{ListIndex!ListIndex(long size)}\\
Constructor of the ListIndex class.

\begin{tcolorbox}[width=\textwidth,myArgs,tabularx={ll|R}]
long & size & Initial size of the list (default value DEFAULT\_stack\_inc).
\end{tcolorbox}

This constructor allocates the default memory for an instance of the List class.
If the size of the list is not specified, the default size is taken into account, which is defined by the value of DEFAULT\_stack\_size.
%@END

\subsection{Basic methods}

%@DOC:ListIndex::insert(Type object, long index)
% Automatic documentation generated from the DynELA source code
% Do not change anything in this LaTeX file between the @DOC and the @END keywords.
\textcolor{purple}{\textbf{ListIndex::insert(Type object, long index)}}\label{ListIndex::insert(Type object, long index)}\index[DL]{ListIndex!insert(Type object, long index)}\\
Insert an object in the list.

\begin{tcolorbox}[width=\textwidth,myArgs,tabularx={ll|R}]
Type & object & Object to insert into the list.\\
long & index & Defines the position of the insertion in the list..
\end{tcolorbox}

This method inserts an object to the list. The object is inserted at a given index in the list, and the list size is automatically incremented if necessary.
%@END

%@DOC:ListIndex::sorted()
% Automatic documentation generated from the DynELA source code
% Do not change anything in this LaTeX file between the @DOC and the @END keywords.
\textcolor{purple}{\textbf{ListIndex::sorted(~)}}\label{ListIndex::sorted()}\index[DL]{ListIndex!sorted(~)}\\
test if the list is sorted.\vspace*{-0.5em}
\begin{tcolorbox}[grow to left by=-1cm, width=\textwidth-1cm,myArgs,tabularx={l|R}]
$\hookrightarrow$ bool&
\end{tcolorbox}

A sorted list is a list where all object have an increasing internal number.
%@END

%@DOC:ListIndex::compacted()
% Automatic documentation generated from the DynELA source code
% Do not change anything in this LaTeX file between the @DOC and the @END keywords.
\textcolor{purple}{\textbf{ListIndex::compacted(~)}}\label{ListIndex::compacted()}\index[DL]{ListIndex!compacted(~)}\\
test if the list is compacted.\vspace*{-0.5em}
\begin{tcolorbox}[grow to left by=-1cm, width=\textwidth-1cm,myArgs,tabularx={l|R}]
$\hookrightarrow$ bool&
\end{tcolorbox}

A sorted list is a list where all object have a continuous increasing internal number.
%@END

%@DOC:ListIndex::flush()
% Automatic documentation generated from the DynELA source code
% Do not change anything in this LaTeX file between the @DOC and the @END keywords.
\textcolor{purple}{\textbf{ListIndex::flush(~)}}\label{ListIndex::flush()}\index[DL]{ListIndex!flush(~)}\\
Empties the list.

This method empties the contents of the stack and returns its real size to zero and its stack size to DEFAULT\_stack\_size.
The stack is as good as new !!! (it's a rechargeable battery !)
%@END

%@DOC:ListIndex::add(Type object)
% Automatic documentation generated from the DynELA source code
% Do not change anything in this LaTeX file between the @DOC and the @END keywords.
\textcolor{purple}{\textbf{ListIndex::add(Type object)}}\label{ListIndex::add(Type object)}\index[DL]{ListIndex!add(Type object)}\\
Add an object to the list.

\begin{tcolorbox}[width=\textwidth,myArgs,tabularx={ll|R}]
Type & object & Object to add to the list.
\end{tcolorbox}

This method adds an object to the list. The object is added to the end of the list, and the list size is automatically incremented if necessary.
%@END

%@DOC:ListIndex::forceSort()
% Automatic documentation generated from the DynELA source code
% Do not change anything in this LaTeX file between the @DOC and the @END keywords.
\textcolor{purple}{\textbf{ListIndex::forceSort(~)}}\label{ListIndex::forceSort()}\index[DL]{ListIndex!forceSort(~)}\\
Sort the list.

%@END

%@DOC:ListIndex::sort()
% Automatic documentation generated from the DynELA source code
% Do not change anything in this LaTeX file between the @DOC and the @END keywords.
\textcolor{purple}{\textbf{ListIndex::sort(~)}}\label{ListIndex::sort()}\index[DL]{ListIndex!sort(~)}\\
Sort the list.

If the list is already sorted this does nothing.
%@END

%@DOC:ListIndex::sort(bool (*f)(Type, Type))
% Automatic documentation generated from the DynELA source code
% Do not change anything in this LaTeX file between the @DOC and the @END keywords.
\textcolor{purple}{\textbf{ListIndex::sort(bool ($\star$f)(Type, Type))}}\label{ListIndex::sort(bool (*f)(Type, Type))}\index[DL]{ListIndex!sort(bool ($\star$f)(Type, Type))}\\
Sort the list using a comparison function.

\begin{tcolorbox}[width=\textwidth,myArgs,tabularx={ll|R}]
bool & *(f) & Function defined in the objects to sort the elements.
\end{tcolorbox}

This method sorts the elements of the stack using a user-defined comparison function.
This method is very powerful for sorting a list and very flexible in use.
The usage may seem complex, but it is defined in the example below.
\begin{CppListing}
class truc
{
  public:
  double z; // a value
};
List <truc*> listeTrucs; // the list
bool compare(truc* p1,truc* p2) // the comparing function
{
  return (p1->z < p2->z); // comparison
}
...
{
...
listeTrucs.sort(compare); // sorts the list using the comparison function
}
\end{CppListing}
%@END

%@DOC:ListIndex::search(long (*f)(Type, long), long i)
% Automatic documentation generated from the DynELA source code
% Do not change anything in this LaTeX file between the @DOC and the @END keywords.
\textcolor{purple}{\textbf{ListIndex::search(long ($\star$f)(Type, long), long i)}}\label{ListIndex::search(long (*f)(Type, long), long i)}\index[DL]{ListIndex!search(long ($\star$f)(Type, long), long i)}\\
Sort the list using a comparison function.

\begin{tcolorbox}[width=\textwidth,myArgs,tabularx={ll|R}]
long & (*f) & Function defined in the objects to sort the elements.\\
long & i & Index of the object
\end{tcolorbox}

This method is used to search for an item in the list using a dichotomous algorithm. This method returns the corresponding element in the list or the NULL value if the element is not in the list.
The usage may seem complex, but it is defined in the example below.
\begin{CppListing}
 class truc
{
  public:
  long z; // a value
};
ListIndex <truc*> listeTrucs; // the list
long compare(truc* p1, long in) // the comparing function
{
  return (p1->z - in); // comparison
}
...
{
...
listeTrucs.sort(compare,10); // seeks for the value 10
}
\end{CppListing}
%@END

%@DOC:ListIndex::del(long start, long stop)
% Automatic documentation generated from the DynELA source code
% Do not change anything in this LaTeX file between the @DOC and the @END keywords.
\textcolor{purple}{\textbf{ListIndex::del(long start, long stop)}}\label{ListIndex::del(long start, long stop)}\index[DL]{ListIndex!del(long start, long stop)}\\
Removes a set of elements from the List.

\begin{tcolorbox}[width=\textwidth,myArgs,tabularx={ll|R}]
long & start & First element to suppress from the list.\\
long & stop & Last element to suppress from the list.
\end{tcolorbox}

This method removes a set of items from the list.
This method is used to remove an entire segment from the list, by defining the start and end indexes of the segment in the list.
If the start and stop parameters are equal, only one element is deleted.
%@END

%@DOC:ListIndex::delBefore(long index)
% Automatic documentation generated from the DynELA source code
% Do not change anything in this LaTeX file between the @DOC and the @END keywords.
\textcolor{purple}{\textbf{ListIndex::delBefore(long index)}}\label{ListIndex::delBefore(long index)}\index[DL]{ListIndex!delBefore(long index)}\\
Removes a set of elements from the List.

\begin{tcolorbox}[width=\textwidth,myArgs,tabularx={ll|R}]
long & index & Index of the last element to suppress from the list.
\end{tcolorbox}

This method removes all items in the list between the beginning of the list and the value given as an argument to this method.
This method is equivalent to del(0, index-1).
%@END

