% !TeX spellcheck = en_US
% !TeX root = DynELA.tex
%
% LaTeX source file of DynELA FEM Code
%
% (c) by Olivier Pantalé 2020
%
\chapter{DynELA Elements library}

\startcontents[chapters]
\printmyminitoc[2]\LETTRINE{T}he \DynELA~is an Explicit FEM code written in \Cpp~using a Python's interface for creating the Finite Element Models. 


\section{Nodes}

%@DOC:Node::Node
% Automatic documentation generated from the DynELA source code
% Do not change anything in this LaTeX file between the @DOC and the @END keywords.
\textcolor{purple}{\textbf{Node::Node}}\label{Node::Node}\index[DL]{Node!Node}\\
Finite Element Node class.

\begin{tcolorbox}[width=\textwidth,myArgs,tabularx={ll|R}]
long & number & Identification number of the node.\\
Vec3D & coords & Coordinates of the corresponding node.\\
Vec3D & disp & Displacement at the current node $\overrightarrow{d}$.\\
double & mass & Nodal mass.\\
List<Element *> & elements & List of the elements that contains a reference to the current node.\\
NodalField* & field0 & Nodal field of the node at the begining of the increment.\\
NodalField* & field1 & Nodal field of the node at the end of the increment.\\
BoundaryCondition* & boundary & Boundary conditions on the current node.
\end{tcolorbox}

This class is used to store information for DynELA Nodes.
%@END

%@DOC:Node::Node(long n, double x, double y, double z)
% Automatic documentation generated from the DynELA source code
% Do not change anything in this LaTeX file between the @DOC and the @END keywords.
\textcolor{purple}{\textbf{Node::Node(long n, double x, double y, double z)}}\label{Node::Node(long n, double x, double y, double z)}\index[DL]{Node!Node(long n, double x, double y, double z)}\\
Constructor of the Node class with initialization.\vspace*{-0.5em}
\begin{tcolorbox}[grow to left by=-1cm, width=\textwidth-1cm,myArgs,tabularx={l|R}]
$\hookrightarrow$ Node & 
\end{tcolorbox}

\begin{tcolorbox}[width=\textwidth,myArgs,tabularx={ll|R}]
long&n&Node number to create.\\
double&x&X coordinate of the node to create.\\
double&y&Y coordinate of the node to create.\\
double&z&Z coordinate of the node to create.
\end{tcolorbox}

%@END

\subsection{Basic methods}

%@DOC:Node::swapFields()
% Automatic documentation generated from the DynELA source code
% Do not change anything in this LaTeX file between the @DOC and the @END keywords.
\textcolor{purple}{\textbf{Node::swapFields(~)}}\label{Node::swapFields()}\index[DL]{Node!swapFields(~)}\\
Swap the two fields of the node.

This method swaps the two fields of the node. So that the field0 becomes the field1 and vice versa.
%@END

%@DOC:Node::fieldScalar(short field)
% Automatic documentation generated from the DynELA source code
% Do not change anything in this LaTeX file between the @DOC and the @END keywords.
\textcolor{purple}{\textbf{Node::fieldScalar(short field)}}\label{Node::fieldScalar(short field)}\index[DL]{Node!fieldScalar(short field)}\\
Get back a nodalField value.\vspace*{-0.5em}
\begin{tcolorbox}[grow to left by=-1cm, width=\textwidth-1cm,myArgs,tabularx={l|R}]
$\hookrightarrow$ double & Nodal field.
\end{tcolorbox}

\begin{tcolorbox}[width=\textwidth,myArgs,tabularx={ll|R}]
short&field&Field to extract (see NodalField for informations).
\end{tcolorbox}

%@END

%@DOC:Node::fieldVec3D(short field)
% Automatic documentation generated from the DynELA source code
% Do not change anything in this LaTeX file between the @DOC and the @END keywords.
\textcolor{purple}{\textbf{Node::fieldVec3D(short field)}}\label{Node::fieldVec3D(short field)}\index[DL]{Node!fieldVec3D(short field)}\\
Get back a nodalField Vec3D.\vspace*{-0.5em}
\begin{tcolorbox}[grow to left by=-1cm, width=\textwidth-1cm,myArgs,tabularx={l|R}]
$\hookrightarrow$ Vec3D & Nodal field.
\end{tcolorbox}

\begin{tcolorbox}[width=\textwidth,myArgs,tabularx={ll|R}]
short&field&Field to extract (see NodalField for informations).
\end{tcolorbox}

%@END


%@DOC:Node::fieldSymTensor2(short field)
% Automatic documentation generated from the DynELA source code
% Do not change anything in this LaTeX file between the @DOC and the @END keywords.
\textcolor{purple}{\textbf{Node::fieldSymTensor2(short field)}}\label{Node::fieldSymTensor2(short field)}\index[DL]{Node!fieldSymTensor2(short field)}\\
Get back a nodalField SymTensor2.\vspace*{-0.5em}
\begin{tcolorbox}[grow to left by=-1cm, width=\textwidth-1cm,myArgs,tabularx={l|R}]
$\hookrightarrow$ SymTensor2 & Nodal field.
\end{tcolorbox}

\begin{tcolorbox}[width=\textwidth,myArgs,tabularx={ll|R}]
short&field&Field to extract (see NodalField for informations).
\end{tcolorbox}

%@END

%@DOC:Node::fieldTensor2(short field)
% Automatic documentation generated from the DynELA source code
% Do not change anything in this LaTeX file between the @DOC and the @END keywords.
\textcolor{purple}{\textbf{Node::fieldTensor2(short field)}}\label{Node::fieldTensor2(short field)}\index[DL]{Node!fieldTensor2(short field)}\\
Get back a nodalField Tensor2.\vspace*{-0.5em}
\begin{tcolorbox}[grow to left by=-1cm, width=\textwidth-1cm,myArgs,tabularx={l|R}]
$\hookrightarrow$ Tensor2 & Nodal field.
\end{tcolorbox}

\begin{tcolorbox}[width=\textwidth,myArgs,tabularx={ll|R}]
short&field&Field to extract (see NodalField for informations).
\end{tcolorbox}

%@END

\section{Elements}

%@DOC:Element::Element
% Automatic documentation generated from the DynELA source code
% Do not change anything in this LaTeX file between the @DOC and the @END keywords.
\textcolor{purple}{\textbf{Element::Element}}\label{Element::Element}\index[DL]{Element!Element}\\
Finite Element Element class.

\begin{tcolorbox}[width=\textwidth,myArgs,tabularx={ll|R}]
long & number & Integration points of the element.\\
ListIndex<Node*> & nodes & Nodes of the element.\\
List<IntegrationPoint*> & integrationPoints & Integration points of the element.\\
List<UnderIntegrationPoint*> & underIntegrationPoints & Under-integration points of the element.\\
Material & material & Material associated to the element.\\
Matrix & stiffnessMatrix & Stiffness matrix of the element.
\end{tcolorbox}

This class is used to store information for DynELA Elements.
The type of element can be one of the following:
\begin{itemize}
\item ElQua4N2D
\item ElTri3N2D
\item ElQua4NAx
\item ElHex8N3D
\item ElTet4N3D
\item ElTet10N3D
\end{itemize}
%@END

%@DOC:Element::Element(long n, double x, double y, double z)
% Automatic documentation generated from the DynELA source code
% Do not change anything in this LaTeX file between the @DOC and the @END keywords.
\textcolor{purple}{\textbf{Element::Element(long n, double x, double y, double z)}}\label{Element::Element(long n, double x, double y, double z)}\index[DL]{Element!Element(long n, double x, double y, double z)}\\
Constructor of the Element class with initialization.\vspace*{-0.5em}
\begin{tcolorbox}[grow to left by=-1cm, width=\textwidth-1cm,myArgs,tabularx={l|R}]
$\hookrightarrow$ Element & 
\end{tcolorbox}

\begin{tcolorbox}[width=\textwidth,myArgs,tabularx={ll|R}]
long & n & Element number to create.
\end{tcolorbox}

%@END

\section{Elements library}

\subsection{Planar elements}

\subsubsection{ElQua4N2D element}

%@DOC:ElQua4N2D::getCharacteristicLength()
% Automatic documentation generated from the DynELA source code
% Do not change anything in this LaTeX file between the @DOC and the @END keywords.
\textcolor{purple}{\textbf{ElQua4N2D::getCharacteristicLength(~)}}\label{ElQua4N2D::getCharacteristicLength()}\index[DL]{ElQua4N2D!getCharacteristicLength(~)}\\
Computation of the characteristic length of the element.\vspace*{-0.5em}
\begin{tcolorbox}[grow to left by=-1cm, width=\textwidth-1cm,myArgs,tabularx={l|R}]
$\hookrightarrow$ double & Characteristic length of the element
\end{tcolorbox}

This method computes the characteristic length of the element from the definition of the geometry of this element.
The relationship used for this calculation is given by:
\begin{equation}
L=\frac{x_{31} y_{42}+x_{24} y_{31}}{\sqrt{x_{24}^2+y_{42}^2+x_{31}^2+y_{31}^2}}
\end{equation}
where $x_{ij}$ is the horizontal distance between points $i$ and $j$ and $y_{ij}$ is the vertical distance between points $i$ and $j$.
%@END

\subsubsection{ElTri3N2D element}

\subsection{Axi-symmetric elements}

\subsubsection{ElQua4NAx element}

%@DOC:ElQua4NAx::getCharacteristicLength()
% Automatic documentation generated from the DynELA source code
% Do not change anything in this LaTeX file between the @DOC and the @END keywords.
\textcolor{purple}{\textbf{ElQua4NAx::getCharacteristicLength(~)}}\label{ElQua4NAx::getCharacteristicLength()}\index[DL]{ElQua4NAx!getCharacteristicLength(~)}\\
Computation of the characteristic length of the element.\vspace*{-0.5em}
\begin{tcolorbox}[grow to left by=-1cm, width=\textwidth-1cm,myArgs,tabularx={l|R}]
$\hookrightarrow$ double & Characteristic length of the element
\end{tcolorbox}

This method computes the characteristic length of the element from the definition of the geometry of this element.
The relationship used for this calculation is given by:
\begin{equation}
L=\frac{x_{31} y_{42}+x_{24} y_{31}}{\sqrt{x_{24}^2+y_{42}^2+x_{31}^2+y_{31}^2}}
\end{equation}
where $x_{ij}$ is the horizontal distance between points $i$ and $j$ and $y_{ij}$ is the vertical distance between points $i$ and $j$.
%@END

\subsection{Three-dimensional elements}

\subsubsection{ElHex8N3D element}

\subsubsection{ElTet4N3D element}

\subsubsection{ElTet10N3D element}
