% !TeX spellcheck = en_US
% !TeX root = DynELA.tex
%
% LaTeX source file of DynELA FEM Code
%
% (c) by Olivier Pantalé 2020
%
\chapter{DynELA Elements library}

\startcontents[chapters]
\printmyminitoc[2]\LETTRINE{T}he \DynELA~is an Explicit FEM code written in \Cpp~using a Python's interface for creating the Finite Element Models. 


\section{Nodes}

\subsection{Constructors}
%@DOC:Node::Node
%Warning :
%This area is an automatic documentation generated from the DynELA source code.
%Do not change anything in this latex file between this position and the @END keyword.
\textcolor{purple}{\textbf{Node::Node}}\label{Node::Node}\index[DL]{Node!Node}\\
Finite Element Node class.

\begin{tcolorbox}[width=\textwidth,myArgs,tabularx={ll|R}]
BoundaryCondition&*boundary&Boundary conditions on the current node.\\
double&mass&Nodal mass.\\
List<Element *>&elements&List of the elements that contains a reference to the current node.\\
long&number&Identification number of the node.\\
NodalField&*currentField&Nodal field of the node.\\
NodalField&*newField&Nodal field of the node.\\
Vec3D&coordinates&Coordinates of the corresponding node.\\
Vec3D&displacement&Displacement at the current node $\overrightarrow{d}$.
\end{tcolorbox}

This class is used to store information for Finite Element Nodes.
%@END

%@DOC:Node::Node(long n, double x, double y, double z)
%Warning :
%This area is an automatic documentation generated from the DynELA source code.
%Do not change anything in this latex file between this position and the @END keyword.
\textcolor{purple}{\textbf{Node::Node(long n, double x, double y, double z)}}\label{Node::Node(long n, double x, double y, double z)}\index[DL]{Node!Node(long n, double x, double y, double z)}\\
Constructor of the Node class with initialization.\\ \hspace*{10mm}$\hookrightarrow$ Node

\begin{tcolorbox}[width=\textwidth,myArgs,tabularx={ll|R}]
long&n&Node number to create.\\
double&x&X coordinate of the node to create.\\
double&y&Y coordinate of the node to create.\\
double&z&Z coordinate of the node to create.
\end{tcolorbox}

%@END

%@DOC:Node::Node(Node node)
%Warning :
%This area is an automatic documentation generated from the DynELA source code.
%Do not change anything in this latex file between this position and the @END keyword.
\textcolor{purple}{\textbf{Node::Node(Node node)}}\label{Node::Node(Node node)}\index[DL]{Node!Node(Node node)}\\
Copy constructor of the Node class.\\ \hspace*{10mm}$\hookrightarrow$ Node

\begin{tcolorbox}[width=\textwidth,myArgs,tabularx={ll|R}]
Node&node&Node to copy.
\end{tcolorbox}

%@END

\subsection{Basic methods}

%@DOC:Node::swapNodalFields()
%Warning :
%This area is an automatic documentation generated from the DynELA source code.
%Do not change anything in this latex file between this position and the @END keyword.
\textcolor{purple}{\textbf{Node::swapNodalFields(~)}}\label{Node::swapNodalFields()}\index[DL]{Node!swapNodalFields(~)}\\
Swap the two nodalFields of the node.

This method swaps the two nodalField of the element. So that the currentField becomes the newField and vice versa.
%@END

%@DOC:Node::copyNodalFieldToNew()
%Warning :
%This area is an automatic documentation generated from the DynELA source code.
%Do not change anything in this latex file between this position and the @END keyword.
\textcolor{purple}{\textbf{Node::copyNodalFieldToNew(~)}}\label{Node::copyNodalFieldToNew()}\index[DL]{Node!copyNodalFieldToNew(~)}\\
Copy the content of the nodalField to the newField.

%@END

%@DOC:Node::getNodalValue(short field)
%Warning :
%This area is an automatic documentation generated from the DynELA source code.
%Do not change anything in this latex file between this position and the @END keyword.
\textcolor{purple}{\textbf{Node::getNodalValue(short field)}}\label{Node::getNodalValue(short field)}\index[DL]{Node!getNodalValue(short field)}\\
Get back a nodalField value.\\ \hspace*{10mm}$\hookrightarrow$ double

\begin{tcolorbox}[width=\textwidth,myArgs,tabularx={ll|R}]
short&field&Field to extract (see NodalField for informations).
\end{tcolorbox}

%@END

%@DOC:Node::getNodalVec3D(short field)
%Warning :
%This area is an automatic documentation generated from the DynELA source code.
%Do not change anything in this latex file between this position and the @END keyword.
\textcolor{purple}{\textbf{Node::getNodalVec3D(short field)}}\label{Node::getNodalVec3D(short field)}\index[DL]{Node!getNodalVec3D(short field)}\\
Get back a nodalField Vec3D.\\ \hspace*{10mm}$\hookrightarrow$ Vec3D

\begin{tcolorbox}[width=\textwidth,myArgs,tabularx={ll|R}]
short&field&Field to extract (see NodalField for informations).
\end{tcolorbox}

%@END


%@DOC:Node::getNodalSymTensor(short field)
%Warning :
%This area is an automatic documentation generated from the DynELA source code.
%Do not change anything in this latex file between this position and the @END keyword.
\textcolor{purple}{\textbf{Node::getNodalSymTensor(short field)}}\label{Node::getNodalSymTensor(short field)}\index[DL]{Node!getNodalSymTensor(short field)}\\
Get back a nodalField SymTensor2.\\ \hspace*{10mm}$\hookrightarrow$ SymTensor2

\begin{tcolorbox}[width=\textwidth,myArgs,tabularx={ll|R}]
short&field&Field to extract (see NodalField for informations).
\end{tcolorbox}

%@END

%@DOC:Node::getNodalTensor(short field)
%Warning :
%This area is an automatic documentation generated from the DynELA source code.
%Do not change anything in this latex file between this position and the @END keyword.
\textcolor{purple}{\textbf{Node::getNodalTensor(short field)}}\label{Node::getNodalTensor(short field)}\index[DL]{Node!getNodalTensor(short field)}\\
Get back a nodalField Tensor2.\\ \hspace*{10mm}$\hookrightarrow$ Tensor2

\begin{tcolorbox}[width=\textwidth,myArgs,tabularx={ll|R}]
short&field&Field to extract (see NodalField for informations).
\end{tcolorbox}

%@END

\section{Planar elements}

\subsection{ElQua4N2D element}

%@DOC:ElQua4N2D::getCharacteristicLength()
%Warning :
%This area is an automatic documentation generated from the DynELA source code.
%Do not change anything in this latex file between this position and the @END keyword.
\textcolor{purple}{\textbf{ElQua4N2D::getCharacteristicLength(~)}}\label{ElQua4N2D::getCharacteristicLength()}\index[DL]{ElQua4N2D!getCharacteristicLength(~)}\\
Computation of the characteristic length of an element.\\ \hspace*{10mm}$\hookrightarrow$ double

This method computes the characteristic length of an element from the definition of the geometry of this element.
The relationship used for this calculation is given by:
\begin{equation}
L=\frac{x_{31} y_{42}+x_{24} y_{31}}{\sqrt{x_{24}^2+y_{42}^2+x_{31}^2+y_{31}^2}}
\end{equation}
where $x_{ij}$ is the horizontal distance between points $i$ and $j$ and $y_{ij}$ is the vertical distance between points $i$ and $j$.
%@END

\subsection{ElTri3N2D element}

\section{Axi-symmetric elements}

\subsection{ElQua4NAx element}

%@DOC:ElQua4NAx::getCharacteristicLength()
%Warning :
%This area is an automatic documentation generated from the DynELA source code.
%Do not change anything in this latex file between this position and the @END keyword.
\textcolor{purple}{\textbf{ElQua4NAx::getCharacteristicLength(~)}}\label{ElQua4NAx::getCharacteristicLength()}\index[DL]{ElQua4NAx!getCharacteristicLength(~)}\\
Computation of the characteristic length of an element.\\ \hspace*{10mm}$\hookrightarrow$ double

This method computes the characteristic length of an element from the definition of the geometry of this element.
The relationship used for this calculation is given by:
\begin{equation}
L=\frac{x_{31} y_{42}+x_{24} y_{31}}{\sqrt{x_{24}^2+y_{42}^2+x_{31}^2+y_{31}^2}}
\end{equation}
where $x_{ij}$ is the horizontal distance between points $i$ and $j$ and $y_{ij}$ is the vertical distance between points $i$ and $j$.
%@END

\section{Three-dimensional elements}

\subsection{ElHex8N3D element}

\subsection{ElTet4N3D element}

\subsection{ElTet10N3D element}
